\clearpage
\section{Conclusion and perspectives}

The main objective of this five months internship is to perform a morpho-kinematical analysis of intermediate redshift ($0.4 < z < 1.4$) galaxies located in low density environments, with [OII] detection, and observed with MUSE in the COSMOS area. The first part consisted in checking the robustness and consistency of morphological parameters we retrieved from HST catalogues, as well as understanding the origin of any discrepancies. We used two criteria, a bias corrected half-light radius and a SNR threshold, to select a sample of $\sim 100$ galaxies which are spatially resolved in our MUSE data. After exploring our sample in terms of redshift and mass-SFR, we cleaned the velocities maps of pixels with fake detection. We modelled the galaxies velocity fields assuming a thin rotating disc, and using a ramp model which takes into account beam-smearing effects, the LSF and the inclination . Using the maximum rotation velocity $V_{\rm{max}}$ and the velocity dispersion $\sigma_{\rm{v}}$ returned by the model, we provide a first morpho-kinematics analysis of our sample. We find a $V_{\rm{max}} / \sigma_{\rm{v}}$ distribution dominated by rotationnaly supported galaxies. We argue that this is most probably due to selection effects and compare it against the results from another survey and a simulation. We also investigate the Tully-Fisher relation, and do not find any evolution with redshift. Finally, we find a correlation between the $\rm{SFR}$ and the velocity dispersion, and we interpret it as an effect of energy injection into the gas during star-formation.\\

Only four months have passed since the beginning. Though the main goal has been achieved, the data analysis is not over yet. The next (and last) month of my internship will be dedicated to this task. The different results provided here and their interpretation will be further investigated, especially the TFR and the origin of its offset with respect to other studies. In the short term these results will also be compared against those obtained by V. Abril Melgajero on the group and cluster galaxies observed in the same fields. But this is only possible if we carefully take into account the different selection criteria used. Nevertheless, we should be able to explore the effect of the environment on the morphological and kinematical properties of galaxies at intermediate redshift and on low mass galaxies as well. 

In the longer term, that is if this is to be continued into a PhD, there is substantial work to do. In terms of methodology, there is the question of the sample selection. By using less restrictive criteria we could almost double its size, but we need to check in more details how these will affect our conclusions. A larger sample also requires more time for the kinematical modelling. This is also an aspect which could be improved by using more reliable values of the galaxies centres. Another important aspect which has not been covered here is the impact of the kinematical model on our conclusions. Other models exist, some based on mass distributions, which would most certainly yield slightly different values of $V_{\rm{max}}$ and $\sigma_{\rm{v}}$. Therefore, the question is: how much different would they be ? In terms of data analysis, we will have only started to answer a few questions about galaxy evolution through cosmic time at the end of the internship. But there are plenty more, from studying the dark matter content of these galaxies (scaling between dynamical and stellar masses across cosmic time ?), to investigating the angular momentum redistribution between dark matter haloes and baryons, as well as understanding the impact of merging and gravitational interaction between galaxies on their morphology and kinematics. 