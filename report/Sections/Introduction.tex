\clearpage
\section{Introduction}
\label{sec:Intro}

\subsection{MUSE-VLT}
\label{subsec:MUSE-VLT}

Photometric and spectroscopic data have been widely used in the past years to study questions related to the evolution of galaxies both at low and high redshifts (insert citation here). Such techniques present both advantages and drawbacks, generally yielding different, complementary information. \\

Assuming the galaxies are resolved enough, photometry gives us information about the morphology of the galaxies and their stellar distribution, allowing one to derive parameters such as the galaxy light profile, its Sérsic index or its half-light radius (cite paper here). Combinations of parameters can then be used to classify the galaxies as elliptical or spiral (respectively early and late type galaxies). If many observations have been carried out on the same sky patch in multiple bands, SED fitting techniques can also be used to derive other useful estimates such as the stellar mass, mean metallicity or star formation rate. Scaling relations between these parameters can then yield new information on the formation scenarios and physical processes at work, as well as constraining the current models of galaxy formation. However, such methods require long observations of the same sky area with different telescopes probing different wavelength ranges. Moreover, the derived parameters are quite sensitive to the redshift of the observed structure which, for photometric redshifts, can have large error bars. \\

On the other hand, astronomers have also been using spectroscopy to study galaxy formation models. Contrary to photometry, this technique does not yield an image of the galaxy but a spectrum of the pointed area. This gives us in return useful information about both the kinematics of the gas we observe and its chemistry. From such spectra, assuming we unambiguously detect a ray, we can infer a much more precise value for the galaxy redshift than in the photometry case as well as the gas velocity and dispersion values. Such data gives us information about how the gas moves inside the galaxies and allows one to classify the galaxies based on their kinematics (rotationally supported or dispersion dominated, see insert citation here). \\

In this context, it has become clear that a combination of these techniques would be necessary to better constrain the models of galaxy formation and to study in more details the origin of the observed scaling laws (SFR-mass, mass-size, luminosity-mass relations, etc.). In recent years, this has become possible with the advent of Integral Field Spectroscopy (IFS) astronomy. This technique combines the advantages of both photometry and spectroscopy by measuring the spectrum of each pixel in an image, yielding a 3D cube with two spatial and one spectral dimensions. Such instruments lacked for the most part a good spatial resolution until recently, but have now filled the gap. \\

MUSE is an IFS instrument mounted on the VLT in Chile which comprises X Integral Field Units (IFU) on a $x \times y \ \ \rm{deg}^2$ area. 
