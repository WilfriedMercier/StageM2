\clearpage
\section{Introduction}
\label{sec:Intro}

\subsection{Current topics in galactic astronomy}

\subsubsection{A global picture of the Universe}

Our knowledge of galaxy properties and their evolution through time has drastically changed from the early 20th century when the Grand Debate between Harlow Shapley and Heber Curtis on the extragalactic or galactic origin of the so-called nebulae took place. We now know, since Hubble first measurement of these nebulae distance to us, that they actually are galaxies like our own lying at great distances and moving away from us. This global movement was understood in the cosmological framework of an expanding Universe. More recent measurements of the radial velocity of far-distant galaxies (cite here) held evidence for an accelerating expansion, which revived the need of a cosmological constant in Einstein's equations of General Relativity. This cosmological constant is interpreted as a fluid with negative pressure acting against gravity, and is known as dark energy. Additionally, the first evidence of a dark component in the galaxy clusters mass distribution was suggested by Fritz Zwicky in 1933. By measuring the relative velocity of $7$ galaxies in Coma Berenices, he derived a velocity dispersion $\sigma_{\rm{v}}$ from which he computed with the virial theorem an order of magnitude for the dynamical mass $M_{\rm{d}} \sim (\sigma_{\rm{v}}^2 R ) / G$ with $R$ the cluster characteristic size. The obtained mass was orders of magnitude above the luminous mass computed from the total luminosity of the galaxies. Though, this calculation was the first evidence for dark matter, it never really convinced other astronomers at the time. A stronger proof for the existence of a dark component embedded in galaxies came from the study of galaxy rotation curves performed by Vera Rubin in the 1970s.


\subsection{Photometry and spectroscopy}
\label{subsec:diffPhotSpec}

\subsubsection{Differences between spectroscopy and photometry}
\label{subsubsec:photo_data}

Studying questions related to the formation and evolution of galaxies requires large datasets of objects at low and high redshifts. The two type of observation which can be carried out to answer these questions are photometry and spectroscopy. These methods were originally independent, yielding unique information on the different components of the galaxies, their distribution, the morphology and the kinematics of these galaxies.
Photometry yields 2D images of galaxies whose surface brightness, that is its total flux divided by its solid angle on the sky, is above the surface brightness limit imposed by the instrument limitations, but also whose size is above the instrument resolution. Depending on the band used, we can observe different galactic components, ranging from the cold, hot and ionised gas to young/old stellar populations, as well as dust.
On the other side, spectroscopy give us information on the strength of the components in the total emitted light by looking at continuum and emission lines. Galaxy spectra can contain high continuum values coming from the combination of old and young stellar populations spread through the galaxy. Detected emission lines, such as $\rm{H}{\alpha}$, $\rm{H}{\beta}$, [OII] and [OIII] doublets or $\rm{Ly}\alpha$ give us insight into the presence of gas and its ionisation, which can in turn indicate the presence of star-forming regions within the galaxy such as [HII] regions. On the contrary, absorption lines tell us about the existence of gas on the line of sight which can be either outside of the galaxy, or directly inside (self-extinction). \\
      
Having 2D images from photometry, it seems natural to derive morphological parameters, such as the ellipticity $e = 1 - b/a$ with $b$ and $a$ the minor and major axes respectively, or a measure of some characteristic radius. Generally, it is assumed that the galaxies light profile has high degrees of symmetry. This implies that a galaxy with an inclination\footnote{We define the inclination as the angle between the plane of the galaxy and the plane of the sky, or equivalently between the normal to the plane of the galaxy and the line of sight.} $i = 0 \degree$ should have circular isophotes (lines of identical surface brightness). When elliptical isophotes are seen, which implies an elliptical shape for the galaxy, this is an indication of a non-zero inclination. Based on our definition, this translates in terms of ellipticity as

\begin{equation}
	\cos i = 1 - e
	\label{eq:inclinaison}
\end{equation}

This assumption is invalid in reality as galaxies have more complex morphologies with central bars, spiral arms, satellite galaxies for the largest ones and potential inflows and/or outflows. It is also incorrect when studying galaxies which have clear disturbed morphologies due to past interactions such as merging events.  \\

In the past, measuring radii and ellipticity were generally done by hand with a method using the curve of growth and which consisted in deriving the size and ellipticity of ellipses at different isophote levels. Nowadays, it has become common to fit galaxy light profile models onto images and to recover the morphological parameters from the best fit. One of the most commonly used light profile is a Sérsic profile which we generally write as \shortcite{PetroRadandMag}\footnote{The original definition $I(r) \propto e^{(r/\alpha)^{1/n}}$ from \shortciteA{Sersic1963} was modified to Eq.\,\ref{eq:SersicProfile} because of the too small (immeasurable) values $\alpha$ generally takes.} 

\begin{equation}
    \label{eq:SersicProfile}
    I(r) = I_{\rm{e}} e^{ - b_n \left ( \left  ( \frac{r}{R_{\rm{e}}}  \right )^{1/n} - 1 \right ) }
\end{equation}

where $r$ is the radial distance to the morphological centre of the galaxy, $n$ is referred as the Sérsic index of the galaxy, $R_e$ is the effective or half-light radius which encloses 50\% of the total luminosity of the galaxy, $I_e$ is the intensity at the position $R_e$ and $b_n$ is a term which ensures that $R_e$ does enclose half the total luminosity. The formal definition of $b_n$ can be shown to be such that $2 \gamma (2n, b_n) = \Gamma (2n)$ with $\gamma$ and $\Gamma$ respectively the incomplete and complete gamma functions.

This equation has been widely used in different context because of its ability to recover two famous profiles:

\begin{itemize}
    \item an exponential disc for $n = 1$ which represents a disk-like/spiral\footnote{We will preferentially use the terms disk-like/late-type galaxies, with respect the Hubble sequence \shortcite{Hubble1922},  \shortcite{Hubble1926}, rather than spiral galaxies since many of the galaxies studied in the present work do show a disk morphology without clear spiral arm patterns.} galaxy. In this case, $r$ generally represents the radial distance in the galactic plane. Sometimes, two exponential discs are used to represents the vertical variation of the emitted light as well as in the plane of the galaxy. 
    \item a de Vaucouleurs profile for $n = 4$ which describes elliptical (early-type) galaxies
\end{itemize}

Depending on the author, only a single Sérsic profile with a freely varying Sérsic index can be used, as is the case for softwares such as SExtractor \shortcite{SExtractor} or GIMD2D \shortcite{GIM2D}. Sometimes, others prefer to use a combination of a bulge and a disk components, with two fixed Sérsic indices. This is the case in GALFIT \shortcite{GALFIT}. \\

Studying galaxy spectra can also be quite a challenge. These can have a noisy continuum emission which can make difficult the detection of emission lines. In general, astronomers look at specific spectral features such as Balmer or Lyman lines, or doublets to improve the confidence in the detection. However, spectroscopes only work in a certain spectral domain, and therefore any line falling out the instrumental range cannot be detected. Given the usual relation between the observed and emitted wavelengths $\lambda_{\rm{obs}} = \lambda{\rm{em}} ( 1+ z)$ with $z$ the redshift of the galaxy, observing some line in a limited wavelength range implies that galaxies can only be detected in a certain redshift interval. \\

Detecting such lines is done through line fitting on a continuum subtracted spectrum. Many algorithms exist with slight variations but, generally, visible lines are fitted one after another with Gaussian shaped profiles. Since all of the emission lines of a given object are redshifted by the same amount, it is possible, based on the relative positions of the different detected lines, to know from which species it comes from and therefore to deduce the redshift of the object. In some cases, it is even possible to detect sources whose spectra are blended on the line of sight. Once the lines are detected, these can be studied individually to derive useful spectral parameters. For instance, measuring the total flux and determining how it compares to the continuum level allows one to derive a Signal to Noise Ratio ($\rm{SNR}$), which can in turn be used when performing statistics on the data. \\

In theory, lines should be intrinsically infinitely thin, but in practice they have a certain width which is directly related to the properties of the emitting species and the physical processes at work in its environment. Assuming Local Thermodynamical Equilibrium (LTE) conditions are met, local variations of velocity due to temperature fluctuations will result in a Gaussian thermal Doppler profile. Turbulent motion will add another Gaussian term because of the variations of the mean velocity of the particles on the line of sight. On the other hand, natural and collision-driven widening of the rays both produce Lorentzian line profiles. In practice, all these processes occur at the same time but with different strength. This results in a ray which is made of a convolution of Gaussian and Lorentzian shaped profiles, which is known as a Voigt profile. \\

For some surveys of the sky such as the Sloan Digital Sky Survey (SDSS), photometric data is available in multiple bands. These surveys combine photometric and spectroscopic data, though the spectral resolution in this case is very low because of the few bands used and their wavelength range. For some objects, spectroscopic follow ups are performed, so that we have both precise photometric and spectroscopic data. However, these kind of surveys with follow ups, have two major flaws. First, it takes double time of observation for the same object as photometry and spectroscopy are performed with two different instruments. The other flaw is that there are only a handful of galaxies with spectroscopic follow ups since not all the objects can be observed. This induces to select objects based on astronomers' interests, which can in turn induces biases in the observed properties.

\subsubsection{Spectroscopy}


\subsubsection{MUSE-VLT}
\label{subsubsec:MUSE-VLT}

MUSE is an Integral Field Unit (IFU) mounted on the VLT in Chile which spans a $\SI{1}{\arcmin} \times \SI{1}{\arcmin}$ Field of View (FoV). Its wavelength range covers both the visual spectrum and the Near Infra-Red part (NIR), going from $\SI{4650}{\angstrom}$ to $\SI{9300}{\angstrom}$. This instrument was built with the main purpose of performing blind searches of sources in the field. The wavelength range is well suited to detect the OII doublet in the redshift range $0.4$, $1.4$.