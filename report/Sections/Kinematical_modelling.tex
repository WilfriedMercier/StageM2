\clearpage
\section{Kinematical modelling}
\label{sec:kinematical_modelling}

\subsection{Manual cleaning}

For every selected galaxy, we ran the automatic cleaning routine as described in Section \ref{sec:selecting_galaxies}. Cleaned maps were produced and the velocity maps were visually inspected. Isolated pixels were removed even if they had a clearly visible [OII] doublet. We also inspected their line near the edges and we removed any pixel whose velocity seemed inconsistent with that of its neighbours (a single negative value around positives ones for instance). Twelve galaxies were unresolved and $3$ were too close to the edges of the MUSE fields which resulted in missing data. Relaxing the $\rm{SNR}$ threshold in the routine to $3$ allowed us to recover $6$ galaxies out of $12$. 

\subsection{Kinematical parametrisation}

The MUSE velocity maps only give us information about a single projected component of the 3D velocity field. Indeed, for a given pixel, the mean offset of the best fit Gaussian profile of the observed line is used in the usual Doppler shift formula to derive the velocity

\begin{equation}
	\frac{\lambda_{\rm{obs}} - \lambda_{\rm{em}}}{\lambda_{\rm{em}}} = \frac{v}{c}
\end{equation}
with $v$ positive if it recedes from us and negative if it approaches. This means the only measure we have is the radial component of the gas velocity within the pixels. If we use the same coordinate system as defined in \shortciteA{Epinat2008}, we can write the observed projected radial component $V_{\rm{obs}}$ as 

\begin{equation}
	V_{\rm{obs}} ( (x_c , y_c) , R, \theta , i, V_{\rm{sys}} , \rm{PA} ) = V_{\rm{sys}} + \left [ V_{\rm{rot}} (R) \cos \theta + V_{\rm{exp}} (R) \sin \theta \right ] \sin i + V_z (R) \cos i
	\label{eq:v_obs}
\end{equation}
where $(x_c , y_c)$ is the galaxy centre position, $R$ the radial distance in the galactic plane, $i$ the galaxy inclination and $\rm{PA}$ the kinematical position angle (which might be different from the morphological one). In Eq.\,\ref{eq:v_obs}, the velocity components are
\begin{enumerate*}[label={(\alph*)}]
	\item $V_{\rm{sys}}$ the systematic velocity due to the expansion of the Universe
	\item $V_{\rm{rot}}$ the rotational velocity tangent to the plane of the galaxy
	\item $V_{\rm{exp}}$ the expansion velocity oriented in the radial direction in the galactic plane
	\item $V_z$ the vertical velocity normal to the plane.
\end{enumerate*}

In practice, spectroscopic measurements give us $\Delta V_{\rm{obs}} = V_{\rm{obs}} - V_{\rm{sys}}$. In Eq.\,\ref{eq:v_obs}, the dependence on the kinematical major axis position angle is implicit since $\theta$ is defined as the angle in the galactic plane starting from the major axis. Thus, $\theta = 0 \degree$ ($90 \degree$) when we measure the velocity of a point on the major (minor) axis. The dependence on the galaxy centre position is also implicit as the radial distance is counted from this point. \\

Both $R$ and $\theta$  can be easily derived given $(x_c , y_c)$ and the $\rm{PA}$, but the inclination of the galaxy must be know in advance. Indeed, models tend to poorly converge when the inclination is let free to vary. More importantly, assuming the rotation curve along the major axis reaches a maximum rotational velocity or at least a plateau, we deduce from Eq.\,\ref{eq:v_obs} that the maximum rotational velocity will scale as

\begin{equation}
	V_{\rm{obs}}^{\rm{max}} (R_{\rm{max}}, i) = V_{\rm{rot}} (R_{\rm{max}}) \sin i
\end{equation}
where $R_{\rm{max}}$ represents the radius at which the maximum/plateau velocity is reached. Thus, just by looking at the rotation curve and by measuring its maximum value, we cannot raise the degeneracy between the inclination, which will lower the velocity, and the real maximum rotational velocity. Hence, it must be used as a fixed input.


\subsection{Model}
\label{subsec:model}

We deduce the kinematical parameters of our galaxies by fitting 2D velocity models onto the cleaned velocity maps. Though a few different models exist, some derived assuming a certain mass distribution, others purely based on visual inspections of the velocity curves, we focussed on a single model. We shall inspect how our conclusions might change if we use other models in future analysis. \\

This model we are using in this analysis consists in describing the rotation curve along the kinematical major axis as a linearly increasing slope in the central part, till reaching a turnover point whereafter the velocity remains constant. This relies on the assumption of a thin disk, which might no longer be true for nearly edge-on galaxies ($75 \degree \lesssim i \lesssim 90 \degree$) due to the increasing depth leading to higher values of opacity. The second assumption is that there actually is a global rotation in our galaxies. In the local Universe, we know this might not be true for ellipticals. However, in our case, we expect spheroidal systems to still have some rotation, though it might be less significant.

We also do not take into account features such as a central bar, arms or clumps. The main reason is that we are lacking resolution to clearly see these effects in the kinematical maps. Some galaxies do show arms and clumpy components in HST images. However, even if the nebular emissions trace these structures, the MUSE PSF will blend them in a manner that all relevant information will be lost. Moreover, these structures are instability driven and the most distant galaxies in our sample are probably still unrelaxed systems. So, in this regard, we expect to see few of them at these redshifts.\\

In our case, we assume we have smooth velocity fields with no bar, no arms and no clumps such as HII regions.
In our model, $4$ morphological and $1$ kinematical parameters are used. The inclination of the galaxy, its centre right ascension and declination and the $PA$ of the major axis must be given as fixed values. The kinematical parameter corresponds to the systematic velocity, that is the radial component of the velocity due to the expansion of the Universe.

\subsection{Maximum velocity and velocity dispersion}

The model returns the minimum, maximum, average and median values for the dispersion map once corrected of beam smearing effects. We also recover the rotational velocity on the plateau $V_{\rm{p}}$, as well as the radius at which it is reached $R_{\rm{p}}$ and the maximum radius where the fit was performed $R_{\rm{last}}$. If the model converges to $R_{\rm{p}} > R_{\rm{last}}$, the measure of both $\rm{R_p}$ and $V_{\rm{p}}$ are not reliable any more. Thus any galaxy meeting this condition is not considered in the following.

To remain consistent with other studies, we decided to compute the maximum velocity at $2.2 R_{\rm{d}}$, with $R_{\rm{d}}$ the disc scale length. This is defined as the radial distance in a morphological exponential disc model where the intensity drops by a factor of $e$ with respect to the central value. From this definition, we can compute its scaling relation with $R_{1/2}$. Given that $I(R_{\rm{d}}) = I_0 e^{-1} = I_{\rm{e}} e^{- b_1 ( R_{\rm{d}} / R_{\rm{e}} - 1 )}$, we find

\begin{equation}
	R_{\rm{e}} = b_1 R_{\rm{d}}
\end{equation}
where $b_1$ is the solution of the equation $2 \gamma (2, b_1) = \Gamma(2)$. This also translates as

\begin{equation}
	\int_0^{b_1} t e^{-t} dt = 1 - e^{-b_1} ( 1 + b_1) = 1/2
\end{equation}
for which we find a solution by making the substitution $b_1 \rightarrow - y - 1$ and using the Lambert W function

\begin{equation}
	b_1 = - \rm{W} \left ( - \frac{1}{2 e} \right ) - 1 \approx 1.67835
\end{equation}
where we only kept the positive, physically meaningful solution. 
