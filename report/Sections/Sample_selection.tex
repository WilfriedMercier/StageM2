\clearpage
\section{Sample selection}
\label{sec:Sample_selection}

\subsection{MUSE-GTO MAGIC}

MAGIC (MUSE-gAlaxy Groups In Cosmos) is  an ESO spectroscopic survey (PI: T. Contini) of intermediate redshift (0.25 < z <  0.85) groups and clusters performed over 4.5 years (Dec 2014 - May 2019) as part of the MUSE Guaranteed Time Observations (GTO). About 100h of MUSE observing time has been used to cover a sample of 15 groups/clusters selected primarily in the COSMOS area following the group catalogue of \shortcite{Knobel2009} and \shortcite{Knobel2012}. The main objective of this survey is to investigate the impact of environment on the evolution of galaxy properties (SFR, kinematics, metallicity, etc) at intermediate redshifts. 

\subsection{COSMOS field}

\begin{table}[htbp]

	\hspace{50pt}

	\begin{tabular}{ccccccc}
	\hline
	Group & Ra\textsuperscript{2} & Dec\textsuperscript{3} & Exposure\textsuperscript{4}  & Average & Total nb. & Nb. field \\
	
	ID\textsuperscript{1} & J2000 (\degree) & J2000 (\degree) & (hr) & seeing\textsuperscript{5} (") & galaxies\textsuperscript{6} & galaxies\textsuperscript{7} \\	
	
	
	\hline
	CGr32 & $149.92194$ & $2.520972$ & $3 \times 4.35$ & $0.51$ - $0.58$ & $277$ & $113$ \\
	\hline
	CGr34\_d & $149.87766$ & $2.502331$ & $5.25$ & $0.63$ & $74$ & $46$ \\
	\hline
	CGr34\_bs & $149.87766$ & $2.502331$ & $4.75$ & $0.58$ & $74$ & $46$ \\
	\hline
	CGr30\_d & $150.144225$ & $2.065971$ & $9.75$ & $0.67$ & $124$ & $65$ \\
	\hline
	CGr30\_bs & $150.144225$ & $2.065971$ & $6.25$ & $0.57$ & $124$ & $65$ \\
	\hline
	CGr84 & $150.057219$ & $2.599744$ & $5.25$ & $0.59$ & $92$ & $51$ \\
	\hline
	CGr84-N & $150.05967$ & $2.61275$ & $1$ & $0.51$ & $81$ & $38$ \\
	\hline
	CGr114 & $149.994285$ & $2.258044$ & $2.2$ & $0.68$ & $64$ & $45$ \\
	\hline
	CGr79 & $149.820686$ & $1.821825$ & $4.35$ & $0.60$ & $65$ & $37$ \\
	\hline
	CGr28 & $150.218094$ & $1.812667$ & $1$ & $0.62$ & $49$ & $29$ \\
	\hline
	CGr26 & $150.492767$ & $2.069139$ & $1$ & $0.59$ & $46$ & $31$ \\
	\hline
	CGr61 & $149.728741$ & $1.915987$ & $1$ & $0.64$ & $43$ & $29$ \\
	\hline
	CGr51 & $149.982756$ & $1.801899$ & $1$ & $0.6-0.7$ & $44$ & $29$ \\
	\hline
	CGr23 & $149.790782$ & $2.162648$ & $1$ & $0.68$ & $51$ & $32$ \\
	\hline
	
	\end{tabular}
	
	\caption[Main characteristics of the COSMOS area.]{\label{table:MUSEfieldsProp}Main characteristics of the COSMOS area. Groups ending with \_d correspond to deep observations (full stacked OBs) and with \_bs correspond to best-seeing observations (only OBs with a seeing below $\SI{0.7}{"}$). The seeing is given for the [OII] wavelength at the group's average redshift. 1. MUSE group number, 2. Group centre's right ascension, 3. Group centre's declination, 4. Length of observations, 5. Average seeing during observation, 6. Total number of detected galaxies within MUSE FoV, 7. Number of field galaxies found by the FoF algorithm.}
\end{table}

The point of the analysis is to perform a joint study of the morphology and the kinematics of field galaxies in the COSMOS area \shortcite{Scoville2007} using respectively HST ACS images and MUSE data. To this end, a set of $12$ galaxy structures (groups or clusters) was selected. The choice of the COSMOS area for this analysis was made because of the large number of multi-band photometric data available for the galaxies in this field and the presence of rich galaxy groups.\\

Guaranteed Time Observations (GTO) centred on the groups were performed from which 14 different MUSE Fields of View (FoV) of $1 \times \SI{1}{arcmin^2} $ were obtained. Each is composed of multiple Observation Blocks (OB) of $40-\SI{60}{min}$ each with the Position Angle (PA) of the instrument rotated by $\SI{90}{\degree}$ between consecutive exposures.

Most of the groups enter into one MUSE FoV, except for CGr32. Since this group is more extended than the others, three slightly overlapping FoVs were taken around it. A couple of groups were also split into \textit{deep} and \textit{best-seeing} observations, the former combining all the OBs regardless of the average seeing in each OB, when the latter only kept OBs with an average seeing lower than $\SI{0.7}{"}$.

 The main characteristics of the observed FoVs, including the position of their centre, the exposure per FoV, the average seeing during the observation, the total number of galaxies and the number of field galaxies detected by the FoF algorithm are listed in Table\,\ref{table:MUSEfieldsProp}. \\

These structures were chosen from \shortcite{Knobel2009}, \shortcite{Knobel2012} catalogues. This ensured them to have both a large set of corresponding photometric data available from \shortciteA{laigle_cosmos2015_2016} catalogue and HST images with a much better resolution (\SI{0.03}{arcsec/px} for HST and $\approx \SI{0.2}{arcsec/px}$ for MUSE). A few galaxies in CGr30\_deep and around some stars might have also been detected within the data cubes but not in HST images. In the former case, the reason is that a blind source detection was performed with ORIGIN \shortcite{Bacon2017} which can deblend sources even below the PSF. For the latter, this is because galaxies were detected in areas around stars which were masked when creating \shortciteA{laigle_cosmos2015_2016} catalogue.

\subsection{Prior information on the galaxies}

\subsubsection{Galaxies in structures}

This internship was planned to be similar in many aspects to what is doing V. Abril-Melgarejo, a 2nd year PhD student supervised by B. Epinat at LAM, Marseille. She studies the morphology and the kinematics of the galaxies within the structures observed with MUSE in the same fields as those we are using in this work. The main difference is that we are focusing on "field" galaxies when she only studies those in structures. To differentiate between group and field galaxies, a Friend of Friends algorithm (FoF) was run prior to my arrival on the galaxies in the MUSE fields. Thus, each galaxy was labelled either as belonging to a structure or as a "field" galaxies. Additionnaly, a morphological analysis had already been performed by V. Abril-Melgarejo with GALFIT on galaxies in structures only. Two Sérsic profiles with fixed Sérsic indices ($n = 1, 4$) were used to describe these galaxies as a combination of a disk and a bulge component. Hence, their intensity can be written as

\begin{equation}
	I(r) = I_{e, \rm{d}} \, e^{-b_1 \left [ \frac{r}{R_{\rm{d}}} -1 \right ]} + I_{e , \rm{b}} e^{- b_4 \left [ \left ( \frac{r}{R_{\rm{b}}} \right )^{1/4} -1 \right ]}
	\label{eq:GALFIT_light_profile}
\end{equation}
where $I_{e, \rm{d}}$, $I_{e, \rm{b}}$ are the effective intensities of the disk and the bulge component respectively and $R_{\rm{d}}$, $R_{\rm{b}}$ their half-light radii.

Therefore, we already have in hand morphological information for roughly half of the total sample including model parameters as described above, but also morphological parameters such as the ellipticity of the galaxies, the position angle of their kinematical main axis (which can be different from the morphological PA).

\subsubsection{Morphological information from COSMOS catalogues}

The total number of galaxies detected in the 14 MUSE fields in the COSMOS area is around $1000$. Roughly half of them belong to structures and the other half are labelled as "field" galaxies. However, not all of them are useful to our study. Some may be too close to the edge for detection, others may be too noisy with a low SNR, or even too small, preventing any relevant kinematical modelling. It is thus mandatory to apply a selection on our data set of field galaxies, first to save time for the analysis, but also to reduce uncertainties.

Our goal is to perform a joint study of the morphology and the kinematics of these galaxies. The tools and the models for the kinematical modelling were already developed as they were used by V. Abril-Melgajero. On the other hand, fitting morphological models with software such as GalFit or SExtractor would have required additional time which we did not have. Hopefully for us, morphological modelling had already been performed on the galaxies in the COSMOS field, so we could focus on the kinematical part. \\

Morphological information for all the galaxies in COSMOS can be found in various catalogues and tables\footnote{\url{https://irsa.ipac.caltech.edu/data/COSMOS/tables/morphology/}}. To start with, we decided to use the two most complete catalogues we could find, that of Tasca (maybe citation) and Cassata (maybe citation as well). Both catalogues contain morphological information including the central position of the galaxy, its half-light radius, concentration and asymmetry parameters, ellipticity, PA of the major morphological axis, and so forth for roughly $232 000$ galaxies. The authors obtained morphological information by running SExtractor on HST images of the galaxies in \shortciteA{laigle_cosmos2015_2016} catalogue. \\

Since we already had in hand a catalogue combining spectroscopic information (emission line fluxes, redshift, etc.) from MUSE measurements with photometric data from \shortciteA{laigle_cosmos2015_2016}, we cross-matched it with Cassata's and Tasca's tables to collect their morphological parameters. We cross-matched first with each catalogue separately, and then with both, using the right ascension $\alpha$ and declination $\delta$ of the centre of the galaxies, allowing for a maximum separation between the MUSE source and the closest source within the catalogues of $\SI{1}{arcsec}$ maximum. This procedure was done for structure galaxies as well in order to use them to check the consistency of the catalogues parameters.

\subsection{Checking catalogues values consistency}
\subsubsection{Reasons for checking catalogues values}

\begin{figure}[H]
	\centering
	\begin{minipage}[c]{0.49\linewidth}
		\includegraphics[width=\linewidth]{../Plots/Error_on_inc_versus_b_a.pdf}
		\subcaption{Error on inclination as a function of $b/a$ and its absolute error. Contours of $\Delta (b/a)$ are plotted in black dashed lines with their corresponding value.}
	\end{minipage}
	\hfill
	\begin{minipage}[c]{0.49\linewidth}
		\includegraphics[width=\linewidth]{../Plots/RelError_on_inc_versus_b_a.pdf}
		\subcaption{Error on inclination as a function of $b/a$ and its relative error. Contours of $\epsilon (b/a)$ are plotted in black and red dashed lines with their corresponding value.}
	\end{minipage}
	\caption[Error on inclination as a function of $b/a$ and its error.]{Error on inclination as a function of $b/a$ and its error. Left: as a function of the absolute error on $b/a$ ($\Delta (b/a)$). Right: as a function of the relative error on $b/a$ ($\epsilon ( b/a)$). Red contours correspond to values for which there is a 50\% and 100\% error on $b/a$.}
	\label{fig:erreur_inclinaison}
\end{figure}

As a first step, we must select a sample based on relevant criteria. This is meant to ensure that we have reliable morphological and kinematical parameters and to reduce statistical errors. Given that any kinematical modelling relies on prior morphological information (galaxy centre, ellipticity, $\rm{PA}$), we can only use a combination of values derived from spectral fitting, for instance the $\rm{SNR}$, and from morphological modelling such as a measure of a galaxy radius to select our sample.  

Before this internship, spectral fitting on the integrated spectra of the galaxies had already been done, and we combined our data with morphological information from COSMOS catalogues as discussed in the previous section. Potentially useful morphological information included half-light radii, magnitudes, ratios of minor to major axis ($b/a$) or equivalently a measure of the ellipticity of the galaxies. Nevertheless, using this data without checking first how well it compares to values found in other catalogues and/or derived using different softwares/models could lead to high biases and uncontrolled errors. Thus, before discussing any selection criteria for our sample, we must first assess the reliability of the parameters we are going to use in later sections. \\

Important values to check are the half-light radius, as it will be used to select our sample, the $b/a$ ratio and the $\rm{PA}$ since these are prior information for the kinematical modelling. We also checked that there was a correlation between GALFIT and the catalogues magnitudes. The axes ratio has a crucial importance since it is directly related to the inclination of the galaxy on the sky through Eq.\,\ref{eq:inclinaison}. Given a certain error $\Delta (b/a)$, and using the usual formula for computing the error $\Delta f = | \partial_x f | \Delta x$ of a function $f$, we find for the inclination

\begin{equation}
	\Delta i = \Delta (b/a) \left | \frac{b}{a} \left ( 2 - \frac{b}{a} \right ) \right | ^{-1/2}
\end{equation}

This is illustrated in Fig.\,\ref{fig:erreur_inclinaison} where $\Delta i$ has been plotted as a function of $b/a$ and its error (absolute on the left, relative on the right). Contours of the error on $b/a$ have been over-plotted to show how evolves $\Delta i$ given a fixed error on $b/a$. As expected, the higher the error on $b/a$ the higher the error on $i$. An error as high as 50\% could yield $\Delta i \approx \SI{27}{\degree}$, though this value is reached for $b/a \approx 1$ where the axes ratio is the least constrained by the morphology. A more appropriate error on $b/a$ of 20\% gives a maximum $\Delta i$ slightly above $\SI{10}{\degree}$, which is correct. 

Since the inclination has a strong impact on the galaxy intrinsic maximum rotational velocity, and potentially on the classification of galaxies as rotationnaly supported or dispersion dominated, this indicates us that for any proper kinematical modelling we must check carefully that the values of axes ratios are consistent between catalogues.

\subsubsection{Catalogues used for comparison}

As stated in previous sections, we cross-matched our catalogue of galaxies detected by MUSE in COSMOS with Cassata's and Tasca's, two tables with morphological information for the galaxies listed in \shortciteA{laigle_cosmos2015_2016}. However, as can be seen in Fig.\,\ref{fig:comp_radii} and \ref{fig:comp_radii_with_bulge_or_disk_radius}, we found large discrepancies between the parameters. Thus, to better understand their origin, we chose to cross-match our catalogue with another one also based on Laigle catalogue. This table has fewer HST counterparts of MUSE galaxies than in the other two but it contains additional morphological information which we can use for the comparison. In addition to that, we already had GALFIT morphological information on $\sim 500$ group galaxies with strong confidence in their value. Therefore, we chose to compare the data in the three morphological catalogues with that of GALFIT.

\subsubsection{Total magnitudes}

\begin{figure}[htbp]
	\includegraphics[width=\linewidth]{../Plots/catalogMag_against_GalfitMag_corrected.pdf}
	\caption[Comparison between magnitudes]{Comparison between the morphological catalogues magnitudes and that of GALFIT for cluster galaxies. Magnitudes from the catalogues agree well between each other. Left: compared with GALFIT disk magnitude only. The slope is too high and a few points are scattered far from the line. Right: compared with the total GALFIT magnitude as defined in Eq.\,\ref{eq:tot_mag_final_version}. We find a linear relation with a scatter of about $\SI{1.6}{mag}$.}
	\label{fig:comp_mags}
\end{figure}

The first value we can easily compare is the apparent magnitude. Cassata's, Tasca's and Zurich's catalogues provide a measure of the total magnitude derived from fitting with SExtractor a single Sérsic profile with a free Sérsic index $n$ on HST images.

Given that GALFIT galaxies light profile was modelled using two Sérsic profiles with fixed Sérsic indices ($n = 1, 4$), we have two measures of their magnitude: one for the bulge component $m_{\rm{b}}^{\rm{GF}}$, and another for the disk component $m_{\rm{d}}^{\rm{GF}}$. To have a meaningful comparison between magnitudes, we need to compute the GALFIT total magnitude by combining the bulge and the disk components. Both are defined as

\begin{equation}
	m_i^{GF} = -2.5 \log_{10} \left ( F_i^{\rm{GF}} \right ) + \rm{C}
	\label{eq:disk_bulge_lum}
\end{equation}
 where $i = \rm{b, d}$ represents either the bulge or the disk, $F = L/{4 \pi D^2}$ is the flux of the galaxy in some band, $L$ its intrinsic luminosity, $D$ its cosmological luminosity distance to us and $\rm{C}$ a constant depending on the band used.

Considering that the two components have different luminosities but are located at the same distance, we can add the fluxes together. Thus the total GALFIT magnitude can also be written as

\begin{equation}
	m_{\rm{tot}}^{\rm{GF}} = - 2.5 \log_{10} \left ( F_{\rm{b}}^{\rm{GF}}  + F_{\rm{d}}^{\rm{GF}} \right ) + \rm{C}
	\label{eq:tot_mag}
\end{equation}

Inverting Eq.\,\ref{eq:disk_bulge_lum} to get the components flux as a function of their magnitude and inserting it into Eq.\,\ref{eq:tot_mag} yields

\begin{equation}
	m_{\rm{tot}}^{\rm{GF}} = -2.5 \log_{10} \left [ 10^{-\frac{m_{\rm{b}}}{2.5}} + 10^{-\frac{\rm{m_d}}{2.5}} \right ]
	\label{eq:tot_mag_final_version}
\end{equation}

This is the value that should be compared with the three catalogues magnitudes. Fig.\,\ref{fig:comp_mags} shows how these scale with each other and with GALFIT disk magnitude on the left, and the total magnitude from Eq.\,\ref{eq:tot_mag_final_version} on the right. As expected, the catalogues give the same value except for a few points. We see that the total GALFIT magnitude gives a much better, poorly scattered linear relation with the catalogues magnitudes. Even though there is an offset between GALFIT and the catalogues, this is due to using different conventions for the constant term in Eq.\,\ref{eq:disk_bulge_lum}.

The same comparison was done on field galaxies, except we did not have GALFIT magnitudes in this case. We also found a good agreement between the catalogues magnitudes.

\subsubsection{Morphological type classification}
\label{subsubsec:classification}

\begin{wrapfigure}{r}{.6\linewidth}
	\centering
	\includegraphics[width=\linewidth]{{../Plots/comparisonClassTypes}.pdf}
	\caption[Morphological types comparison]{Comparison between morphological types given in Tasca and Zurich catalogues against that of Cassata. Galaxies are labelled as follows: E for ellipticals, Sp for spirals/disks-like, Irr for irregulars. The percentage of galaxies falling into the given classes is indicated in red and the method compared with Cassata's is shown on the top left corner of each plot. We find good agreement between Tasca and Cassata types but not between Cassata and Zurich.}
	\label{fig:morpho_comp}
\end{wrapfigure}

We might expect to have some discrepancies in our data because of the models used between GALFIT and SExtractor/GIM2D. A way to check this effect is to study how these differences scale with the morphological type of the galaxies. For instance, if we use the disk half-light radius of GALFIT to compare with that of SExtractor, we might expect to have some scatter in our relation for the elliptical galaxies as the disk component is not the best one to describe them.\\

To see how these relations scale with morphological types, we can use the classification given in the three morphological catalogues. These classifications are based on methods which can be quite different and which can use morphological parameters in different ways. A more detailed explanation of the parameters used, of how these methods work, of their strength and weaknesses can be found in Appendix \ref{appendix:classification}. We provide below a short introduction to these classifications:

\begin{itemize}
	\item Cassata's catalogue gives a classification based on morphological parameters they derived with SExtractor. To do so, they use a reference of $500$ galaxies with known parameters which they visually classify as either elliptical, disk-like/spiral or irregular. From this set, each time a new galaxy must be classified, its $11$ closest neighbours are inspected and the most frequent class is assigned to the galaxy.
	
	\item Tasca's catalogue gives different classifications based on three methods. The first one is similar to the one used by Cassata. This is also the classification they recommend to use because this is the one they put the more their trust in. The second one uses the technique described in (insert Abraham 1996 here) using the asymmetry and concentration parameters. The last one uses a support vector machine to classify galaxies.
	
	\item Zurich's catalogue gives a single classification called Zurich Estimator of Structural Type (ZEST) which is described in \shortcite{Scarlata2007}. This method is based on a Principal Component Analysis (PCA). They decided to keep the first three Principal Components (PA) which retain most of the information present in the original five parameters (concentration, asymmetry, Gini coefficient, second-order moment of the brightest pixels producing $20\%$ of the total flux and the ellipticity of the galaxy).
\end{itemize}

The morphological types given in Tasca's and Zurich's catalogues are compared against the class given by Cassata in Fig.\,\ref{fig:morpho_comp}. We observe a good agreement between Cassata's and Tasca's types with just a few elliptical galaxies labelled as disk-like and vice versa. Roughly $50\%$ of the galaxies appear to be elliptical. On the contrary, Zurich's classification seems to label more than $70\%$ of the galaxies as disk-like, including a large number of elliptical galaxies .\\

Considering the recommendation of Tasca to use its Int class and since we find a good agreement between Cassata's morphological type and those given by Tasca, we decided to use and to stick to Cassata's class throughout this work whenever we needed to separate galaxies between elliptical/disk-like/irregulars. This choice also ensured us to have the largest sample possible with a coherent classification as Cassata's catalogue is the one with the largest number of HST counterparts of MUSE galaxies in the COSMOS field. Because of the incompatible results between Zurich's and Cassata's/Tasca's types, we decided to put aside these values and not use them, though this shall require further investigation in future work to assess the origin of these discrepancies.

\subsubsection{Half-light radii}
\label{sec:comp_radii}

\begin{figure}[htbp]
	\centering
	\includegraphics[width=\linewidth]{{../Plots/plotsWithColourCoding/relErr_against_galFit1.5LightRadius_colourCoded_matchAllTypes_TwoGalFitRadii}.pdf}
	\caption[Radii comparison between catalogues and GALFIT using bulb and disk]{Relative error on the half-light radius between the catalogues and GALFIT. Points have been colour coded according to their classification given in Cassata's catalogue (Irr for irregular, Sp for spiral/disk, E for ellipticals). Top: GALFIT disk radius is used for all the points. We observe an underestimation of the catalogues radius with respect to that of GALFIT for elliptical galaxies. Bottom: same plot with GALFIT bulge radius used for elliptical galaxies. In this case, we find an overestimation of the radius.}
	\label{fig:comp_radii_with_bulge_or_disk_radius}
\end{figure}

\begin{figure}[htbp]
	\centering
	\includegraphics[width=\linewidth]{{../Plots/plotsWithColourCoding/relErr_against_GalFit1.5LightRadius_colourCoded_CassataType}.pdf}
	\caption[Radii comparison per catalogue]{Comparison between half-light radii from the morphological catalogues and the radius of GALFIT disk component. This plot is similar to Fig.\,\ref{fig:comp_radii_with_bulge_or_disk_radius} top plot but each catalogue was separated in its own subplot. Left: full range of relative error. Right: a zoom on the points with $R_{1/2}^{\rm{GF}} \geq \SI{5}{px}$.}
	\label{fig:comp_radii}
\end{figure}

Perhaps the most important parameter we have to check is the half-light radius. Indeed, if we underestimate it, we might remove from our sample resolved galaxies and therefore reduce our statistics. On the other hand, overestimating it would give us too many unresolved galaxies for which we would spend time removing noise dominated pixels without being able to perform their kinematical analysis in the end. Hence, it is mandatory to thoroughly check the values of the half-light radius from the three catalogues against that of GALFIT, and understand the origin of any discrepancies if there happens to be some. \\

We found a quite large disagreement between GALFIT half-light radius and those given in the morphological catalogues, as well as between them. This is illustrated in Fig.\,\ref{fig:comp_radii_with_bulge_or_disk_radius} and \ref{fig:comp_radii} where half-light radii in the catalogues are compared against that of GALFIT. Galaxies are colour coded according to the classification given in Cassata's catalogue. We checked that using Tasca's classifications as described in Sec.\,\ref{subsubsec:classification} did not change our conclusions. In these plots, we decided to use for the x-axis the half-light radius of the GALFIT disk component $R_{1/2 , \rm{d}}^{\rm{GF}}$ for all the galaxies, even though we might expect the ellipticals to be better described by their GALFIT bulge half-light radius. This choice is further discussed below.\\

The catalogues radii seem to be overestimated with respect to that of GALFIT for low $R_{1/2 , \rm{d}}^{\rm{GF}}$. These are computed with SExtractor and, given that it does not take into account the PSF in its fitting routine, we expect the PSF to dominate for galaxies with small angular sizes. On the contrary, GALFIT does take into account the PSF in its calculation, and therefore we expect its half-light radius to decrease with lower values of $R_{1/2 , \rm{d}}^{\rm{GF}}$. 

When focussing on galaxies with a GALFIT radius larger than the HST-ACS PSF $\rm{FWHM}$ which is around $\SI{0.15}{"}$ ($4 - \SI{5}{px}$), we observe a global underestimation for all the catalogues, up to roughly 50\%. This scatter is mainly due to elliptical galaxies. On the contrary, radii of disk-like galaxies have the least scatter and biais, especially the values given in Zurich's catalogue. This different behaviour between elliptical and disk-like galaxies might be explained, as mentioned above, by the fact we are using the half-light radius of GALFIT disk component to asses the reliability of the elliptical galaxies radii give in the catalogues. A better choice may be to use the bulge component, which better describes the light profile of an elliptical galaxy, and its half-light radius for this population.\\

If we decide to split the galaxies into two categories, ellipticals and disks/irregulars, and if we use for the first category the bulge radius $R_{1/2 , \rm{b}}^{\rm{GF}}$, and for the second $R_{1/2 , \rm{d}}^{\rm{GF}}$, we find that elliptical galaxies half-light radii from the catalogues are now overestimated as shown in Fig.\,\ref{fig:comp_radii_with_bulge_or_disk_radius} lower panel. These results suggest that elliptical galaxies in this sample are neither dominated (in terms of radius) by the disk component, nor by the bulge in the GALFIT model. Even though we did not push further the analysis on this discrepancy, we still mention that the next step should be to directly compute the half-light radius of the GALFIT model by integrating Eq.\,\ref{eq:GALFIT_light_profile} up to half its total luminosity.


\begin{figure}[htbp]
	\centering
	\includegraphics[width=\linewidth]{{../Plots/Selection_plots/correctedRadiusLinearRegression}.pdf}
	\caption[Bias before/after correction]{Biais and scatter before and after applying the correction mentioned in the paragraph. Left: half-light radius relative error as a function of Cassata's radius. Right: scaling between Cassata's and GALFIT radii. The scatter around $\SI{0.2}{px}$ is not significantly changed after the bias correction.}
	\label{fig:correction_biais_radius}
\end{figure}

Based on Fig.\,\ref{fig:comp_radii}, we decided to use the galaxies half-light radii given in Zurich catalogue, keeping in mind that we might have a non-negligible fraction of elliptical galaxies for which their value may be underestimated. However, this is the catalogue with least HST counterparts. Thus, in order to increase our sample, we also chose to select additional galaxies found in Cassata's catalogue but not in Zurich's. As it seems that Cassata's radius is more bias than Zurich's, we fitted a linear relation onto the relative error data, which translates as

\begin{equation}
	R_{1/2}^{\rm{Corrected}} = R_{1/2}^{\rm{Cassata}} \left ( 1 + \beta + \alpha R_{1/2}^{\rm{Cassata}} \right ) ^{-1}
\end{equation}
where $R_{1/2}^{\rm{Corrected}}$ is the new half-light radius after the bias correction and $\alpha$, $\beta$ the parameters of our fit. Based on previous arguments, we only fitted galaxies for which $R_{1/2}^{\rm{GF}} > \SI{5}{px}$. Once a best fit solution was found, we slightly adjusted it by applying weights on ellipticals in order to limit their impact on disk-like galaxies radii which appeared to be too much overestimated. We found $\alpha = 1.95 \times 10^{-3}$ and $\beta = -0.349$, which gives us $261$ field galaxies with a radius either from Zurich or from the corrected value of Cassata.


\subsubsection{Ellipticity}

\begin{figure}[H]
	\centering
	\begin{minipage}[c]{0.49\linewidth}
		\includegraphics[width=\linewidth]{{../Plots/checkMorpho/check_b_a}.pdf}
		\subcaption{Axes ratio from Zurich and Cassata for the field galaxies. There is an overall good agreement between the catalogues and between Zurich and GIM2D. We find a similar dispersion for SExtractor and GIM2D of about $0.1$. No specific trend is found with the galaxies size.}
	\end{minipage}
	\hfill
	\begin{minipage}[c]{0.49\linewidth}
		\includegraphics[width=\linewidth]{{../Plots/checkMorpho/check_b_a_GF}.pdf}
		\subcaption{GALFIT axes ratio against that of Cassata for galaxies in structures. Globally, the axes ratio in the catalogue seems to be underestimated. If we separate the galaxies as disk or bulge dominated according to the most luminous component, we find that most of the scatter comes from elliptical galaxies.}
	\end{minipage}
	\caption[Ellipticity comparison]{Comparison between Cassata, Zurich and GALFIT $b/a$ axes ratio. Zurich catalogue gives two values computed with SExtractor and GIM2D, and Cassata only with SExtractor. Error bars are only available for a few GIM2D and GALFIT values.}
	\label{fig:check_ellipticity}
\end{figure}

As mentioned in previous sections, the ellipticity is a critical parameter which is used to compute a value for the inclination of the galaxies. It is related to the $b/a$ ratio given in the catalogues, so we can compare it as a proxy for the ellipticity. This is illustrated in Fig.\,\ref{fig:check_ellipticity} 
\begin{enumerate*}[label={(\alph*)}]
	\item where the axes ratio is compared for field galaxies for which we could assign a radius as described in Section \ref{sec:comp_radii}
	\item GALFIT $b/a$ is compared with that of Cassata for structure galaxies.
\end{enumerate*}
Catalogues give consistent values for field galaxies with most of the scatter coming from the region where $0.4 < b/a < 0.8$. We checked that the size of the galaxies did not impact too much on the error, but we did not find any specific trend. We also checked that catalogues values are comparable to that of GALFIT. In this case, we find a large scatter with the catalogues axes ratio underestimated with respect to GALFIT value for almost all galaxies. Since GALFIT fits two components on the light profile, we expect the galaxies with a dominant bulge to have a less constrained $b/a$ ratio. To check this effect, we colour coded the points according to the difference in magnitude between GALFIT disk and bulge components $M_{\rm{d}}^{\rm{GF}} - M_{\rm{b}}^{\rm{GF}}$. We observe that the majority of the scatter actually comes from bulge dominated galaxies, i.e. ellipticals, for which $M_{\rm{d}}^{\rm{GF}} - M_{\rm{b}}^{\rm{GF}} > 0$.















\subsection{SNR and size selection}
\label{sec:cut}

\subsubsection{Size selection}
\label{sec:cut_size}

\begin{wrapfigure}{l}{.6\linewidth}
	\centering
	\includegraphics[width=\linewidth]{{../Plots/Selection_plots/SNR_vs_R_halfLight_arcsec}.pdf}
	\caption[Selected sample]{Full sample of $261$ field galaxies with our selection box over-plotted (hatched area). Galaxies with $5 \leq \rm{SNR} \leq 15$ and $0.25" \leq R_{1/2} \leq 0.45"$ (bounds plotted as dashed lines) were visually inspected to check how many resolved galaxies we might have lost. Selection criteria from Sec.\,\ref{sec:cut} are also plotted as solid red lines. No specific trend appear with redshift.}
	\label{fig:sample_selection}
\end{wrapfigure}

The first parameter we can think of to make our selection is the size of the galaxies. We already checked in Section\,\ref{sec:comp_radii} biases which might arise, investigated their origins, and corrected them. Now that we have reliable values, we need to define a selection criterion. Based on \shortciteA{Bacon2015} and \shortciteA{Bacon2017}, the MUSE Point Spread Function (PSF), that is the pattern we obtain when we observe a point-like source with MUSE, can either be described by   a \shortciteA{MoffatProfile} or a Gaussian profile. In practice, they showed that a Gaussian best fits the PSF for MUSE images, so we used this profile in the following parts whenever we needed the PSF. \\

The PSF Full Width at Half Maximum ($\rm{FWHM}$) is directly related to the seeing conditions, and it can be easily derived from the relation $I_{\rm{PSF}} ( \rm{FWHM}/2) = I_0 /2$. Since it gives us information about the minimum spatial extent within which we start to loose information, we could use it as a starting point for our selection criterion. Moreover, we will need a reliable measure of the $\rm{FWHM}$ for each galaxy when performing the kinematical modelling. Indeed, according to the aforementioned articles the value of the $\rm{FWHM}$ is expected to linearly decrease with wavelength. All galaxies are observed via their [OII] $\lambda\lambda 3729, \SI{3729}{\angstrom}$ doublet at the same rest-frame wavelength. But, given that they are all field galaxies located at a different redshift $z$, we actually observe them at wavelengths covering the entire MUSE spectrum, that is we have the usual relation $\lambda_{\rm{obs}} = \lambda_{\rm{em}} ( 1 + z )$, where $\lambda_{\rm{em}}$, $\lambda_{\rm{obs}}$ are the emitted (rest-frame) and observed wavelengths respectively. Therefore, there is not just one $\rm{FWHM}$ value per field, but one per galaxy. \\

To derive this value, we need to compute the linear evolution in each field by measuring the $\rm{FWHM}$ of stars for at least two different wavelengths. Assuming seeing conditions are similar within a field (no spatial dependence), we can use the same relation per MUSE field to compute the $\rm{FWHM}$ for the [OII] wavelength at the redshift of the galaxies. These measures had already been done by B. Epinat and V. Abril-Melgajero on at least two stars per field. \\

Though a more rigorous modelling of the wavelength variation of the PSF $\rm{FWHM}$ including both more data points and potentially higher order terms is mandatory for future analysis, we decided to stick to this values in the present work, keeping in mind the uncertainties which will affect the velocity dispersion maps in the modelling section. A representation of the $\rm{FWHM}$ variation with wavelength for the 16 observations is shown in Fig.\,\ref{fig:FWHM_var_lambda}. Most MUSE fields have $\rm{FWHM}$ values below $\SI{0.7}{"}$ which is not surprising given that it was one of the constraints of the observations. CGr23 is the only FoV to have its $\rm{FWHM}$ above $\SI{0.7}{"}$ for almost every wavelength. Only galaxies further than $z \approx 1.11$. In Table.\,\ref{table:MUSEfieldsProp}, the seeing is the average over the OBs for the [OII] doublet at the redshift of the group which is around $1.17$. Hence the value below $\SI{0.7}{"}$. \\

To have a criterion which is not galaxy dependent, we decided to keep galaxies with $2 R_{1/2} \geq \SI{0.7}{"}$ since this corresponds to an upper limit for the $\rm{FWHM}$ for almost every field. Moreover, according to \shortciteA{Swinbank2017} who compared the half-light radius of the nebular [OII] emission in MUSE images with that of their HST counterpart in ACS \textit{I} or WFC3 \textit{H}-band, the [OII] half-light radius seems to scale with the HST radius as 

\begin{equation}
	R_{1/2}^{\rm{OII}} = (1.18 \pm 0.03) R_{1/2}^{\rm{HST}}
\end{equation}

Thus, by using an upper limit, we might loose a few galaxies which might have been resolved enough. Nevertheless, this was mandatory if we wanted to keep a sample not too large given the time constraints.

\subsubsection{SNR selection}
\label{sec:cut_SNR}

The other information we use to select our sample is the signal to noise ratio. The $\rm{SNR}$ is generally derived as the ratio between the source's signal and the background level. The noisier an image, the lower the SNR is. Given that galaxy pixels dominated by noise will be removed before the modelling by an automatic cleaning routine, we would like to keep galaxies with a high enough $\rm{SNR}$, so that there is still a significant amount of pixels after the cleaning. 

In the MUSE pipeline, \textsc{PLATEFIT} \shortcite{Tremonti2004} was run on the integrated spectrum of the galaxies after deriving their redshift. This software uses a set of stellar spectra from \shortciteA{Bruzual2003} and \shortciteA{Sanchez2006} to fit and remove the continuum emission at the galaxy redshift. Each line is then fitted by a Gaussian profile individually with the same velocity offset and dispersion for all the lines. \textsc{PLATEFIT} returns galaxies spectral parameters such as the [OII] flux. From these, we used the flux and its error to compute a value for the $\rm{SNR}$ as

\begin{equation}
	\rm{SNR} = \frac{\rm{[OII] \,\, flux}}{\rm{[OII] \,\, flux \,\, error}}
\end{equation}

Since the typical $\rm{SNR}$ value used by the routine we run to remove noise dominated pixels in the MUSE maps is around $5$, we decided as a first step to choose an SNR lower limit of $10$, allowing us to keep galaxies with strong enough detection after the automatic cleaning is performed.


\subsection{Characterisation of the sample}
\label{sec:sample_characterisation}

\subsubsection{Selecting galaxies}
\label{sec:selecting_galaxies}

\begin{wrapfigure}{r}{.4\linewidth}
	\centering
	\vspace{-100pt}
	\includegraphics[width=\linewidth]{{../Plots/Selection_plots/cumulative_hist_redshift}.pdf}
	\caption[Redshift distribution]{Redshift distribution of the total sample (grey), selected (green) and unselected (blue) galaxies. Redshift bins of size $0.1$ have been used. Top: density plot as a function of redshift. Bottom: cumulative distribution. We lack most of the galaxies at redshift $1.4$. Other redshift bins do not loose to many galaxies with respect to the unselected ones except maybe at $z \approx 0.5 $.}
	\label{fig:redshift_distribution}
\end{wrapfigure}

We decided to visually inspect galaxies around the $\rm{SNR}$ and the size cuts to quantify
\begin{enumerate*}[label={(\alph*)}]
	\item how many resolved galaxies would be lost if we applied the criteria given in Sec.\,\ref{sec:cut_size} and \ref{sec:cut_SNR} (false negative)
	\item how many unresolved galaxies would be selected in our sample (false positive).
\end{enumerate*}
To do so, we defined four boxes: $5 \leq \rm{SNR} \leq 10$, $10 \le \rm{SNR} \leq 15$, $0.25" \leq R_{1/2} \leq 0.35"$ and $0.35" \le R_{1/2} \leq 0.45"$ containing $46$, $20$, $58$ and $49$ galaxies respectively. \\

We then ran an automatic cleaning routine for all the galaxies in these boxes. This algorithm removes any pixel which either has its $\rm{SNR}$ below $5$ or a dispersion below a certain percentage $\gamma$ of the velocity dispersion $\sigma_{\rm{v}}$ computed from the Line Spread Function (LSF) $\rm{FWHM}$. The LSF corresponds to the spectral equivalent of an image PSF, and tells us how much the instrument will broaden out an infinitely thin line. According to \shortcite{Bacon2017} and \shortcite{Guerou2017} who measured the LSF variation with wavelength in the Hubble Ultra Deep Field (HUDF) and Hubble Deep Field South (HDFS) and who showed that it was stable through time, we computed the LSF $\rm{FWHM}$ as

\begin{equation}
	\rm{FWHM} = a \lambda^2 + b \lambda + c
\end{equation}
with $a = 5.866 \times 10^{-8} \si{\per\angstrom}$, $b = - 9.187$, $c = \SI{6.040}{\angstrom}$ and $\lambda$ the observed wavelength. Thus, the LSF $\rm{FWHM}$ will depend upon the redshift of the galaxy as well. It is related to the velocity dispersion via

\begin{equation}
	\frac{\sigma_{\rm{v}}}{c} = \frac{\sigma}{\lambda_{\rm{em}} (1+z)}
\end{equation}
where $\sigma = \rm{FWHM} \left (2 \sqrt{2 \log 2} \right )^{-1}$ is the spectral dispersion due to the LSF. \\

The choice to keep pixels with a velocity dispersion above $\gamma \sigma_{\rm{v}}$ was motivated by the fact that sky lines from OH molecules in the atmosphere can generate false [OII] emission lines. Sky subtraction during the data reduction can also add new lines in the MUSE images. However, these are always found to have a width below the LSF $\rm{FWHM}$. So any pixel with a false detection will be removed with this choice. The percentage $\gamma$, which we chose to be $80\%$, is meant to not remove pixels with real detection which might have thin lines. \\

After visually inspecting their sizes in the cleaned maps, we classified them as resolved or not. These boxes, the $\rm{SNR}$, half-light radius and redshift of the full sample of field galaxies before selection are shown in Fig.\ref{fig:sample_selection}. Among the $261$ field galaxies, $103$ fall into the limits imposed in Sec.\,\ref{sec:cut}. We find that roughly $26\%$ of the galaxies below the cuts are actually false negatives and that $15\%$ above them are false positives. Since this classification is purely visual, slightly relaxing the constraints on to whether a galaxy is resolved enough or not allow these values to vary by roughly $10\%$. Our criteria for the selection therefore seem fine, though in future works we might increase our sample by a factor of $\sim 1.5$ by visually inspected each galaxy separately.

\subsubsection{Redshift distribution}

The choice of the [OII] $\lambda\lambda 3729, \SI{3729}{\angstrom}$ doublet for the kinematical analysis was due in part to the large range of redshift it covered given the MUSE spectrum. We therefore had in our catalogue galaxies spanning a redshift range from $0.3$ to about $1.4$. As show in Fig.\,\ref{fig:redshift_distribution}, after our selection we loose a significant fraction of galaxies with $z$ around $0.3$, $0.5$ and $0.75$ as well as the majority of the most distant galaxies. It is not surprising to loose galaxies at high redshift as these have the lowest angular sizes on the sky at a fixed physical length. Nevertheless, we are globally selecting less galaxies per redshift bin than we are putting aside. The average and median redshifts are $0.749$ and $0.719$ for the selected galaxies, $0.809$ and $0.750$ for the unselected ones. Thus, our sample remains consistent in terms of redshift even after the selection.


\subsubsection{Mass-SFR relation}

\begin{figure}[hbtp]
	\centering
	\includegraphics[width=\linewidth]{{../Plots/Selection_plots/SFR_vs_mass_withoutFAST}.pdf}
	\caption[SFR-mass relation]{SFR-mass relation for selected (filled circles) and unselected (open circles) galaxies. Left: SFR extracted from the morphological catalogues. Right: the redshift corrected SFR as given in Eq.\,\ref{eq:SFR_corrected_redshift}. We obtain the usual main sequence of galaxies as a linear relation between SFR and mass (in log-log space), as well as quenched galaxies on the lower right part of the plane which are removed when performing the selection.}
	\label{fig:sfr_vs_mass}
\end{figure}

Our sample spans 3 orders of magnitude both in mass and $\rm{SFR}$ in the ranges $10^{8.07} \leq M / \si{M_{\odot}} \leq  10^{11.1}$ and $10^{-1.40} \leq \rm{SFR} / ( \si{M_{\odot}. yr^{-1} }) \leq 10^{2.10}$. We can investigate where our sample lies in the mass-SFR diagram as shown in Fig.\,\ref{fig:sfr_vs_mass}. Star-forming galaxies are expected to lie on a diagonal line with some scatter in $\log M$-$\log \rm{SFR}$ space called the galaxies Main Sequence (MS) or the star-formation sequence. A few highly star-forming galaxies called Ultra Luminous InfraRed Galaxies (ULIRG/LIRG) are generally found above the main sequence. Some massive quiescent (low SFR values) galaxies can also be found below the MS. In our sample, we recover it after selecting the galaxies, but we loose the massive ones with low SFR values. This is probably due to the fact that we are selecting galaxies with a high enough [OII] surface brightness. Since [OII] flux comes from ionised regions which correlate with star formation, most quiescent galaxies are eliminated when observing, and the few large enough to be observed will have too low $\rm{SNR}$ values to be selected.  \\

We also checked if there was any redshift evolution in the main sequence. We refer to \shortciteA{Boogaard2018} who derived a redshift evolution for their galaxies on the MS found roughly in the same redshift, mass and $\rm{SFR}$ intervals. They modelled the log of the $\rm{SFR}$ as a plane in $\log M$ - $\log (1 + z)$ space 

\begin{equation}
	\log_{10} \left ( \frac{\rm{SFR}}{\si{M_{\odot} . yr^{-1}}} \right ) = 0.83_{-0.06}^{+0.07} \log_{10} \left ( \frac{M}{\si{M_{\odot}}} \right ) - 0.83^{+0.05}_{-0.05} + 1.74_{-0.68}^{+0.66} \log_{10} \left ( \frac{1+z}{1+z_0} \right )
	\label{eq:SFR_corrected_redshift}
\end{equation}
where $z_0$ acts as a normalisation factor which will scale the relation at this redshift. We chose the median redshift of the selected galaxies to apply the $\rm{SFR}$ correction shown in the right plot of Fig.\,\ref{fig:sfr_vs_mass}. Though we do not find a significant improvement in the scatter after the correction is applied.



 
