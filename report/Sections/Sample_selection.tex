\clearpage
\section{Sample selection}
\label{sec:Sample_selection}

The point of the analysis is to perform a joint study of the morphology and the kinematics of field galaxies in the COSMOS field using respectively HST ACS images and MUSE data. A set of Guaranteed Time Observations (GTO runs) was performed. Around a dozen MUSE fields were taken, focusing on galaxy clusters and groups in the COSMOS field. A Friend of Friends algorithm (FoF) was run previous to my arrival on the galaxies in the different clusters in order to separate them in two categories: cluster and field galaxies. \\

Generally speaking, galaxies detected by MUSE are also detected with HST imaging because of its much better resolution (0.03" for HST and 0.15" for MUSE). Nevertheless, the MUSE pipeline allows in some cases the detection of sources in regions where there is no HST counterpart. In addition to that, studying the kinematics of the galaxies require them to be resolved enough in the MUSE OII flux maps, which means they must have both a size large enough and a mean Signal to Noise Ratio (SNR) high enough as well. Thus, it is mandatory to perform a selection in our sample of galaxies before performing any study on their parameters. \\

The total number of galaxies detected with MUSE in the COSMOS field is around . The data returned by the MUSE pipeline which was run on the galaxies was cross-matched with the information given in \shortciteA{laigle_cosmos2015_2016} catalogue, which is the most complete one for the COSMOS field. Thus, for each MUSE field, a table containing all the information for each galaxy was built.\\

As a first step before the selection, we combined all the data from the different MUSE fields into two master catalogs, the first one with

