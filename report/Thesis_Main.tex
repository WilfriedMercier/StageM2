\documentclass[a4paper, twoside, 11pt]{article}
 
\usepackage[T1]{fontenc}
 
%bibliography package
\usepackage{multicol}
 
%text packages
\usepackage[utf8]{inputenc}
\usepackage[T1]{fontenc}
\usepackage[inline]{enumitem}
\usepackage{hyperref}
\usepackage{xcolor}

%figure packages
\usepackage{graphicx}
\usepackage{float}
\usepackage{wrapfig}
\usepackage{caption, subcaption}

%table packages
\usepackage{longtable}

%math/physics packages
\usepackage{gensymb}
\usepackage{amsmath}
\usepackage{amssymb}
\usepackage{siunitx}
\usepackage{breakcites}

% set margins for double-sided printing
\usepackage[left=2.5cm, right=2.5cm, top=2.5cm, bottom=2.5cm, head=15pt]{geometry} 
\usepackage{setspace}
\onehalfspacing
% set headers
\usepackage{fancyhdr}
\pagestyle{fancy}
\fancyhead{}
\fancyfoot{}
\fancyhead[LE,RO]{\textsl{\leftmark}}
\fancyhead[RE,LO]{\thesisauthor}
\fancyfoot[C]{\thepage}
\renewcommand{\headrulewidth}{0.4pt}
\renewcommand{\footrulewidth}{0pt}

%page formatting packages
\usepackage{afterpage}
\newcommand\blankpage{%
    \null
    \thispagestyle{empty}%
    \addtocounter{page}{-1}%
    \newpage}

% set APA citation style
\usepackage{apacite}
\usepackage[numbib,notlof,notlot,nottoc]{tocbibind}
\pagenumbering{gobble}

%%%%%%%%%%%%%%%%%%%%%%%%%%%%%%%%%%%%%%%%%%%%%%%%%%%%%%%%%%%%%
%THESIS Parameters 
%%%%%%%%%%%%%%%%%%%%%%%%%%%%%%%%%%%%%%%%%%%%%%%%%%%%%%%%%%%%%

\title{Evolution of galaxy dynamics over the last 10 Gyrs with MUSE/VLT}

\newcommand{\thesisdate}{April 22, 2019}
\newcommand{\thesisauthor}{\textsc{Mercier} Wilfried}
\newcommand{\thesistype}{Master Thesis}
\newcommand{\supervisor}{\textsc{Contini} Thierry}
\newcommand{\cosupervisor}{\textsc{Epinat} Benoît}

%%%%%%%%%%%%%%%%%%%%%%%%%%%%%%%%%%%%%%%%%%%%%%%%%%%%%%%%%%%%%
%DOCUMENT
%%%%%%%%%%%%%%%%%%%%%%%%%%%%%%%%%%%%%%%%%%%%%%%%%%%%%%%%%%%%%

\begin{document}

%%%%%%%%%%%%%%%%%%%%%%%%%%%%%%%%%%%%%%%%%%%%%%%%%%%%%%%%%%%%%
%TITLE PAGE (Pre-defined, just change parameters above)
%%%%%%%%%%%%%%%%%%%%%%%%%%%%%%%%%%%%%%%%%%%%%%%%%%%%%%%%%%%%%
%%%%%%%%%%%%%%%%%%%%%%%%%%%%%%%%%%%%%%%%%%%%%%%%%%%%%%%%%%%%%
%TITLE PAGE
%%%%%%%%%%%%%%%%%%%%%%%%%%%%%%%%%%%%%%%%%%%%%%%%%%%%%%%%%%%%%
\makeatletter
\begin{titlepage}
    \begin{center}
        \vspace*{.5cm}

        \Large
        \textbf{\@title}

        \vspace{1.5cm}
        
        \thesistype{}
        
        \vspace{1cm}

        \begin{figure}[htbp]
             \centering
             \includegraphics[width=.5\linewidth]{./Figures/logo_obspm.png}
        \end{figure}

        \vspace{1cm}

        \large
        \textbf{Author}: \thesisauthor{}\\
        \large
        \textbf{Supervisor}: \supervisor{}\\
        \large
        \textbf{Co-Supervisor}: \cosupervisor{}

        \vspace{1cm}
        \large
        Observatoire de Paris \\
        Institut de Recherche en Astrophysique et Planétologie \\
        Laboratoire d'Astrophysique de Marseille

        \vspace{1cm}
        \@date
        
        \begin{figure}[htbp]
             \centering
             \vspace{.25cm}
             \includegraphics[width=.25\linewidth]{./Figures/psl_logo.png}
             \hspace{35pt}
             \includegraphics[width=.17\linewidth]{./Figures/omp_logo.jpg} \\
             \vspace{.25cm}
             \hspace{35pt}
             \includegraphics[width=.25\linewidth]{./Figures/irap_logo.png}
             \hspace{35pt}
             \includegraphics[width=.25\linewidth]{./Figures/Logo_LAM_petit.jpg}
        \end{figure}

    \end{center}
\end{titlepage}
\makeatother
\afterpage{\blankpage}

%%%%%%%%%%%%%%%%%%%%%%%%%%%%%%%%%%%%%%%%%%%%%%%%%%%%%%%%%%%%%
%ABSTRACT
%%%%%%%%%%%%%%%%%%%%%%%%%%%%%%%%%%%%%%%%%%%%%%%%%%%%%%%%%%%%%
\clearpage
\thispagestyle{empty}
\section*{Abstract}

We present a morpho-kinematics study of $\sim 100$ star-forming galaxies located in low-density environment (so called "field" galaxies) with [OII] detection in the redshift range $0.4 < z < 1.4$ observed with MUSE, and with HST-ACS counterparts, in the COSMOS area. The main goal was to perform the kinematical modelling of a subsample of spatially resolved galaxies in an initial MUSE sample of $\sim 500$ galaxies at intermediate redshift in order to study the joint evolution of their morphological (size, inclination) and kinematical parameters (maximum rotation velocity and velocity dispersion) over a large range of stellar mass ($\sim 10^8 - 10^{11}$) and redshift. 

In Section\,\ref{sec:Intro}, after introducing the field of extra-galactic astrophysics and explaining the importance of morpho-kinematics analysis on the understanding of galaxy evolution through cosmic time, we give a brief overview of Integral Field Spectroscopy concept and MUSE instrument.

The methodology behind the selection of a spatially resolved sample of galaxies is given in Section\,\ref{sec:Sample_selection}. We retrieve from three different HST catalogues the morphological parameters of approximately half the field galaxies. We then check their consistency and robustness against each other, as well as against a morphological modelling performed by V. Abril-Melgarejo (PhD student at LAM, Marseille) on another sample of group and cluster galaxies observed in the same MUSE fields. After correcting for any bias, we select a subsample of $\sim 100$ spatially resolved galaxies with high enough signal-to-noise [OII] detection. According to this selection, we are loosing galaxies at $z \approx 1.4$, as well as those with a low stellar mass (low size) or which are nearly quenched (low [OII] $\rm{SNR}$).

Velocities maps based on the [OII] emission-line doublet, and produced with the CAMEL software, are automatically cleaned of pixels with fake detection and then visually inspected. Remaining isolated pixels and/or with large velocity offsets ($\approx 50 - \SI{100}{\kilo\meter \, \second^{-1}}$) were also removed. We model the galaxies with a thin rotating disc  using a ramp model with $3$ free parameters, from which we derive inclination, beam-smearing and LSF corrected values for the maximum rotation velocity ($V_{\rm{max}}$) and the velocity dispersion ($\sigma_{\rm{v}}$). We give a description of the modelling in Section\,\ref{sec:kinematical_modelling}.

We provide an early investigation of the link between the morphological and kinematical properties of our sample in Section \ref{sec:morpho_kin_analysis}. The $V_{\rm{max}}/ \sigma_{\rm{v}}$ distribution is looked upon. We find that $90\%$ of our sample are rotationally supported galaxies ($V_{\rm{max}}/ \sigma_{\rm{v}} > 2$) and give some insight into why we have a larger proportion than in other studies. We recover the Tully-Fisher Relation and investigate its potential evolution with redshift and galaxy size. A correlation is found between $\rm{SFR}$ and velocity dispersion which seems consistent with an origin of the velocity dispersion mainly driven by energy injection during star formation.

Finally we conclude on the work accomplished during these three months and we give some indications about the next questions which will be investigated during the remaining and last month of internship at IRAP and potentially beyond if this internship is to be continued into a PhD.



%%%%%%%%%%%%%%%%%%%%%%%%%%%%%%%%%%%%%%%%%%%%%%%%%%%%%%%%%%%%%
%TOC,TOF,TOT
%%%%%%%%%%%%%%%%%%%%%%%%%%%%%%%%%%%%%%%%%%%%%%%%%%%%%%%%%%%%%
\clearpage
\pagenumbering{Roman}
\tableofcontents
\clearpage
\listoffigures
\listoftables

\newpage
\section*{Acknowledgements}

Before beginning this report, or finishing it should I say as it is the last bit I am working on, I would like to thank a few people for the opportunity which was given to me to do my Master 2 internship at IRAP. To start with, I would like to greatly thank my two supervisors, Thierry Contini and Benoit Epinat, for their continuous support throughout this internship, for the many corrections given to the report and all the interesting discussions which followed. I would also like to thank them for letting me go to LAM, in Marseille. This stay was a great opportunity to learn more about the kinematics of galaxies observed with integral field spectroscopy while visiting during the evenings a city I had never been to until now. Also a great thank to Valentina for teaching me how to properly use all the different IDL codes and for never hesitating to answer my questions.

 I obviously thank everyone who has been reading and correcting mistakes and typos in the report, starting once again with Thierry and Benoit, but also Lina Issa who has been an attentive reader during this last month. Thanks to the Observatory of Paris and IRAP for allowing me to do this internship, as well as everyone in the GAHEC group.
 
 Finally, I more generally thank everyone I have spent time with at IRAP. The other students from the Observatory of Paris, Aline, Cyril, Naïs, Mikel and Sylvain, Segolène for spending time in the same office and for supporting with me the intense music coming from outside during the last day of the report writing, and all the other interns and PhD students we spent time with during and after lunch.

\newpage
\null\thispagestyle{empty}
\newpage
\pagenumbering{arabic}


%%%%%%%%%%%%%%%%%%%%%%%%%%%%%%%%%%%%%%%%%%%%%%%%%%%%%%%%%%%%%
%MAIN PART
%%%%%%%%%%%%%%%%%%%%%%%%%%%%%%%%%%%%%%%%%%%%%%%%%%%%%%%%%%%%%

% Introduction
\clearpage
\section{Introduction}
\label{sec:Intro}

\subsection{Current topics in galactic astronomy}

\subsubsection{A global picture of the Universe}

Our knowledge of galaxy properties and their evolution through time has drastically changed from the early 20th century when the Grand Debate between Harlow Shapley and Heber Curtis on the extragalactic or galactic origin of the so-called nebulae took place. We now know, since Hubble first measurement of these nebulae distance to us, that they actually are galaxies like our own lying at great distances and moving away from us. This global movement was understood in the cosmological framework of an expanding Universe. More recent measurements of the radial velocity of far-distant galaxies (cite here) held evidence for an accelerating expansion, which revived the need of a cosmological constant in Einstein's equations of General Relativity. This cosmological constant is interpreted as a fluid with negative pressure acting against gravity, and is known as dark energy. Additionally, the first evidence of a dark component in the galaxy clusters mass distribution was suggested by Fritz Zwicky in 1933. By measuring the relative velocity of $7$ galaxies in Coma Berenices, he derived a velocity dispersion $\sigma_{\rm{v}}$ from which he computed with the virial theorem an order of magnitude for the dynamical mass $M_{\rm{d}} \sim (\sigma_{\rm{v}}^2 R ) / G$ with $R$ the cluster characteristic size. The obtained mass was orders of magnitude above the luminous mass computed from the total luminosity of the galaxies. Though, this calculation was the first evidence for dark matter, it never really convinced other astronomers at the time. A stronger proof for the existence of a dark component embedded in galaxies came from the study of galaxy rotation curves performed by Vera Rubin in the 1970s.


\subsection{Photometry and spectroscopy}
\label{subsec:diffPhotSpec}

\subsubsection{Differences between spectroscopy and photometry}
\label{subsubsec:photo_data}

Studying questions related to the formation and evolution of galaxies requires large datasets of objects at low and high redshifts. The two type of observation which can be carried out to answer these questions are photometry and spectroscopy. These methods were originally independent, yielding unique information on the different components of the galaxies, their distribution, the morphology and the kinematics of these galaxies.
Photometry yields 2D images of galaxies whose surface brightness, that is its total flux divided by its solid angle on the sky, is above the surface brightness limit imposed by the instrument limitations, but also whose size is above the instrument resolution. Depending on the band used, we can observe different galactic components, ranging from the cold, hot and ionised gas to young/old stellar populations, as well as dust.
On the other side, spectroscopy give us information on the strength of the components in the total emitted light by looking at continuum and emission lines. Galaxy spectra can contain high continuum values coming from the combination of old and young stellar populations spread through the galaxy. Detected emission lines, such as $\rm{H}{\alpha}$, $\rm{H}{\beta}$, [OII] and [OIII] doublets or $\rm{Ly}\alpha$ give us insight into the presence of gas and its ionisation, which can in turn indicate the presence of star-forming regions within the galaxy such as [HII] regions. On the contrary, absorption lines tell us about the existence of gas on the line of sight which can be either outside of the galaxy, or directly inside (self-extinction). \\
      
Having 2D images from photometry, it seems natural to derive morphological parameters, such as the ellipticity $e = 1 - b/a$ with $b$ and $a$ the minor and major axes respectively, or a measure of some characteristic radius. Generally, it is assumed that the galaxies light profile has high degrees of symmetry. This implies that a galaxy with an inclination\footnote{We define the inclination as the angle between the plane of the galaxy and the plane of the sky, or equivalently between the normal to the plane of the galaxy and the line of sight.} $i = 0 \degree$ should have circular isophotes (lines of identical surface brightness). When elliptical isophotes are seen, which implies an elliptical shape for the galaxy, this is an indication of a non-zero inclination. Based on our definition, this translates in terms of ellipticity as

\begin{equation}
	\cos i = 1 - e
	\label{eq:inclinaison}
\end{equation}

This assumption is invalid in reality as galaxies have more complex morphologies with central bars, spiral arms, satellite galaxies for the largest ones and potential inflows and/or outflows. It is also incorrect when studying galaxies which have clear disturbed morphologies due to past interactions such as merging events.  \\

In the past, measuring radii and ellipticity were generally done by hand with a method using the curve of growth and which consisted in deriving the size and ellipticity of ellipses at different isophote levels. Nowadays, it has become common to fit galaxy light profile models onto images and to recover the morphological parameters from the best fit. One of the most commonly used light profile is a Sérsic profile which we generally write as \shortcite{PetroRadandMag}\footnote{The original definition $I(r) \propto e^{(r/\alpha)^{1/n}}$ from \shortciteA{Sersic1963} was modified to Eq.\,\ref{eq:SersicProfile} because of the too small (immeasurable) values $\alpha$ generally takes.} 

\begin{equation}
    \label{eq:SersicProfile}
    I(r) = I_{\rm{e}} e^{ - b_n \left ( \left  ( \frac{r}{R_{\rm{e}}}  \right )^{1/n} - 1 \right ) }
\end{equation}

where $r$ is the radial distance to the morphological centre of the galaxy, $n$ is referred as the Sérsic index of the galaxy, $R_e$ is the effective or half-light radius which encloses 50\% of the total luminosity of the galaxy, $I_e$ is the intensity at the position $R_e$ and $b_n$ is a term which ensures that $R_e$ does enclose half the total luminosity. The formal definition of $b_n$ can be shown to be such that $2 \gamma (2n, b_n) = \Gamma (2n)$ with $\gamma$ and $\Gamma$ respectively the incomplete and complete gamma functions.

This equation has been widely used in different context because of its ability to recover two famous profiles:

\begin{itemize}
    \item an exponential disc for $n = 1$ which represents a disk-like/spiral\footnote{We will preferentially use the terms disk-like/late-type galaxies, with respect the Hubble sequence \shortcite{Hubble1922},  \shortcite{Hubble1926}, rather than spiral galaxies since many of the galaxies studied in the present work do show a disk morphology without clear spiral arm patterns.} galaxy. In this case, $r$ generally represents the radial distance in the galactic plane. Sometimes, two exponential discs are used to represents the vertical variation of the emitted light as well as in the plane of the galaxy. 
    \item a de Vaucouleurs profile for $n = 4$ which describes elliptical (early-type) galaxies
\end{itemize}

Depending on the author, only a single Sérsic profile with a freely varying Sérsic index can be used, as is the case for softwares such as SExtractor \shortcite{SExtractor} or GIMD2D \shortcite{GIM2D}. Sometimes, others prefer to use a combination of a bulge and a disk components, with two fixed Sérsic indices. This is the case in GALFIT \shortcite{GALFIT}. \\

Studying galaxy spectra can also be quite a challenge. These can have a noisy continuum emission which can make difficult the detection of emission lines. In general, astronomers look at specific spectral features such as Balmer or Lyman lines, or doublets to improve the confidence in the detection. However, spectroscopes only work in a certain spectral domain, and therefore any line falling out the instrumental range cannot be detected. Given the usual relation between the observed and emitted wavelengths $\lambda_{\rm{obs}} = \lambda{\rm{em}} ( 1+ z)$ with $z$ the redshift of the galaxy, observing some line in a limited wavelength range implies that galaxies can only be detected in a certain redshift interval. \\

Detecting such lines is done through line fitting on a continuum subtracted spectrum. Many algorithms exist with slight variations but, generally, visible lines are fitted one after another with Gaussian shaped profiles. Since all of the emission lines of a given object are redshifted by the same amount, it is possible, based on the relative positions of the different detected lines, to know from which species it comes from and therefore to deduce the redshift of the object. In some cases, it is even possible to detect sources whose spectra are blended on the line of sight. Once the lines are detected, these can be studied individually to derive useful spectral parameters. For instance, measuring the total flux and determining how it compares to the continuum level allows one to derive a Signal to Noise Ratio ($\rm{SNR}$), which can in turn be used when performing statistics on the data. \\

In theory, lines should be intrinsically infinitely thin, but in practice they have a certain width which is directly related to the properties of the emitting species and the physical processes at work in its environment. Assuming Local Thermodynamical Equilibrium (LTE) conditions are met, local variations of velocity due to temperature fluctuations will result in a Gaussian thermal Doppler profile. Turbulent motion will add another Gaussian term because of the variations of the mean velocity of the particles on the line of sight. On the other hand, natural and collision-driven widening of the rays both produce Lorentzian line profiles. In practice, all these processes occur at the same time but with different strength. This results in a ray which is made of a convolution of Gaussian and Lorentzian shaped profiles, which is known as a Voigt profile. \\

For some surveys of the sky such as the Sloan Digital Sky Survey (SDSS), photometric data is available in multiple bands. These surveys combine photometric and spectroscopic data, though the spectral resolution in this case is very low because of the few bands used and their wavelength range. For some objects, spectroscopic follow ups are performed, so that we have both precise photometric and spectroscopic data. However, these kind of surveys with follow ups, have two major flaws. First, it takes double time of observation for the same object as photometry and spectroscopy are performed with two different instruments. The other flaw is that there are only a handful of galaxies with spectroscopic follow ups since not all the objects can be observed. This induces to select objects based on astronomers' interests, which can in turn induces biases in the observed properties.

\subsubsection{Spectroscopy}


\subsubsection{MUSE-VLT}
\label{subsubsec:MUSE-VLT}

MUSE is an Integral Field Unit (IFU) mounted on the VLT in Chile which spans a $\SI{1}{\arcmin} \times \SI{1}{\arcmin}$ Field of View (FoV). Its wavelength range covers both the visual spectrum and the Near Infra-Red part (NIR), going from $\SI{4650}{\angstrom}$ to $\SI{9300}{\angstrom}$. This instrument was built with the main purpose of performing blind searches of sources in the field. The wavelength range is well suited to detect the OII doublet in the redshift range $0.4$, $1.4$.

% Sample selection
\clearpage
\section{Sample selection}
\label{sec:Sample_selection}

\subsection{MUSE-GTO MAGIC}

\subsection{COSMOS field}

\begin{table}[htbp]

	\hspace{50pt}

	\begin{tabular}{ccccccc}
	\hline
	Group & Ra\textsuperscript{2} & Dec\textsuperscript{3} & Exposure\textsuperscript{4}  & Average & Total nb. & Nb. field \\
	
	ID\textsuperscript{1} & J2000 (\degree) & J2000 (\degree) & (hr) & seeing\textsuperscript{5} (") & galaxies\textsuperscript{6} & galaxies\textsuperscript{7} \\	
	
	\hline
	\hline
	CGr32 & NaN & NaN & $3 \times 4.35$ & $0.51$ - $0.58$ & NaN & NaN \\
	\hline
	CGr34\_d & $149.87766$ & $2.502331$ & $5.25$ & $0.63$ & NaN & NaN \\
	\hline
	CGr34\_bs & $149.87766$ & $2.502331$ & $4.75$ & NaN & NaN & NaN \\
	\hline
	CGr30\_d & $150.144225$ & $2.065971$ & $9.75$ & $0.67$ & NaN & NaN \\
	\hline
	CGr30\_bs & $150.144225$ & $2.065971$ & $6.25$ & NaN & NaN & NaN \\
	\hline
	CGr84 & $150.057219$ & $2.599744$ & $5.25$ & $0.59$ & NaN & NaN \\
	\hline
	CGr84-N & NaN & NaN & $1$ & $0.51$ & NaN & NaN \\
	\hline
	CGr114 & $149.994285$ & $2.258044$ & $2.2$ & $0.68$ & NaN & NaN \\
	\hline
	CGr79 & $149.820686$ & $1.821825$ & $4.35$ & $0.60$ & NaN & NaN \\
	\hline
	CGr28 & $150.218094$ & $1.812667$ & $1$ & $0.62$ & NaN & NaN \\
	\hline
	CGr26 & $150.492767$ & $2.069139$ & $1$ & $0.59$ & NaN & NaN \\
	\hline
	CGr61 & $149.728741$ & $1.915987$ & $1$ & $0.64$ & NaN & NaN \\
	\hline
	CGr51 & $149.982756$ & $1.801899$ & $1$ & $0.6-0.7$ & NaN & NaN \\
	\hline
	CGr23 & $149.790782$ & $2.162648$ & $1$ & $0.68$ & NaN & NaN \\
	\hline
	
	\end{tabular}
	
	\caption[Main characteristics of the observed MUSE fields]{\label{table:MUSEfieldsProp}Main characteristics of the observed MUSE fields. Groups ending with \_d correspond to deep observations (full stacked OBs) and with \_bs correspond to best-seeing observations (only OBs with a seeing below $\SI{0.7}{"}$). The seeing is given for the [OII] wavelength at the group's redshift. 1. MUSE group number, 2. Group centre's right ascension, 3. Group centre's declination, 4. Duration of observations, 5. Average seeing during observation, 6. Total number of detected galaxies within MUSE FoV, 7. Number of field galaxies found by the FoF algorithm.}
\end{table}

The point of the analysis is to perform a joint study of the morphology and the kinematics of field galaxies in COSMOS \shortcite{Scoville2007} using respectively HST ACS images and MUSE data. To this end, a set of $12$ galaxy structures (these can be either groups or clusters) in COSMOS was selected. The choice of the COSMOS field for this analysis was made because of the large number of multi-band photometric data available for the galaxies in this field and the presence of rich (large number of member galaxies) galaxy groups.\\

Guaranteed Time Observations (GTO) centred on the groups were performed from which 14 different MUSE Fields of View (FoV) of $1 \times \SI{1}{arcmin^2} $ were obtained. Each is composed of Observation Blocks (OB) of \textcolor{red}{$\SI{30}{min}$} each with the Position Angle (PA) of the instrument rotated by $\SI{90}{\degree}$ between consecutive observations.

Most of the groups are in one FoV, except for CGr32. Since this group is \textcolor{red}{larger} than the others, three slightly overlapping FoVs were taken around it. A couple of groups were also split into \textit{deep} and \textit{best-seeing} observations, the former combining all the OBs regardless of the average seeing in each OB, when the latter only kept OBs with an average seeing higher than $\SI{0.7}{"}$.

 The main characteristics of the observed FoVs, including the position of their centre, the exposure per FoV, the average seeing during the observation, the total number of galaxies and the number of field galaxies detected by the FoF algorithm are listed in Table\,\ref{table:MUSEfieldsProp}. \\

These structures were chosen within the COSMOS field. This ensured them to have both a large set of corresponding photometric data available from \shortciteA{laigle_cosmos2015_2016} catalogue and HST images with a much better resolution (\SI{0.03}{arcsec/px} for HST and $\sim \SI{0.2}{arcsec/px}$ for MUSE). A few galaxies in CGr30\_deep and around some stars might have also been detected within the data cubes but not in HST images. In the former case, the reason is that a blind source detection was performed with ORIGIN \shortcite{Bacon2017} which can deblend sources even below the PSF. For the latter, this is because galaxies were detected in areas around stars which were masked when creating \shortciteA{laigle_cosmos2015_2016} catalogue.

\subsection{Prior information on the galaxies}

\subsubsection{Galaxies in structures}

This internship was planned to be similar in many aspects to what has done V. Abril-Melgajero in LAM, Marseille. She studied the morphology and the kinematics of the galaxies within the structures observed by MUSE in COSMOS. The galaxies were therefore found in the same MUSE fields as those we are using in this work, but belong to structures instead of being labelled as \textcolor{red}{field galaxies}. 

To differentiate between group and field galaxies, a Friend of Friends algorithm (FoF) was run prior to my arrival on the galaxies in each MUSE field. Thus, each galaxy was labelled either as belonging to a structure or as field galaxies.

Additionnaly, a morphological analysis had already been performed by V. Abril-Melgajero with GALFIT on galaxies in structures. Two Sérsic profiles with fixed Sérsic indices ($n = 1, 4$) were used to describe these galaxies as a combination of a disk and a bulge component. Hence, their intensity can be written as

\begin{equation}
	I(r) = I_{e, \rm{d}} \, e^{-b_1 \left [ \frac{r}{R_{\rm{d}}} -1 \right ]} + I_{e , \rm{b}} e^{- b_4 \left [ \left ( \frac{r}{R_{\rm{b}}} \right )^{1/4} -1 \right ]}
	\label{eq:GALFIT_light_profile}
\end{equation}
where $I_{e, \rm{d}}$, $I_{e, \rm{b}}$ are the effective intensities of the disk and the bulge component respectively and $R_{\rm{d}}$, $R_{\rm{b}}$ their half-light radii.

Therefore, we already had morphological information for roughly half of the total sample including model parameters as described above, but also morphological parameters such as the ellipticity of the galaxies, the Position Angle (PA) of their kinematical main axis (which can be different from the morphological PA).

\subsubsection{Morphological information from COSMOS catalogues}

The total number of galaxies detected by MUSE in COSMOS is around $1000$. Roughly half of them belong to structures and the other half are labelled as field galaxies. Among these galaxies, not all of them are useful to our study. Some may be too close to the edge, others be too noisy with a low Signal to Noise Ratio (SNR), or too small for any relevant kinematical modelling. It is thus mandatory to apply a selection on our data set of field galaxies, first to save time for the analysis, but also to reduce uncertainties.

Our goal is to perform a joint study of the morphology and the kinematics of these galaxies. The tools and the models for the kinematical modelling were already developed as they were used by V. Abril-Melgajero. On the other hand, fitting morphological models with software such as GalFit or SExtractor would have required additional time which we did not have. Hopefully for us, morphological modelling had already been performed on the galaxies in the COSMOS field, so we could focus on the kinematical part. \\

Morphological information for all the galaxies in COSMOS can be found in various catalogues and tables\footnote{\url{https://irsa.ipac.caltech.edu/data/COSMOS/tables/morphology/}}. To start with, we decided to use the two most complete catalogues we could find, that of Tasca (maybe citation) and Cassata (maybe citation as well). Both catalogues contain morphological information including the central position of the galaxy, its half-light radius, concentration and asymmetry parameters, ellipticity, Position Angle of the major morphological axis (PA), and many more for roughly $232 000$ galaxies. The authors obtained morphological information by running SExtractor on HST images of the galaxies in \shortciteA{laigle_cosmos2015_2016} catalogue. \\

Since we already had data from \shortciteA{laigle_cosmos2015_2016} for our galaxies, we only had to cross-match our table with Cassata's and Tasca's catalogues to collect their morphological parameters. We decided to cross-match our data with each catalogue separately and then with both using the right ascension $\alpha$ and declination $\delta$ of the centre of the galaxies, allowing for a maximum separation between the MUSE source and the closest source within Cassata's and Tasca's catalogues of $\SI{1}{arcsec}$ maximum. 

This cross-matching procedure was performed for both field and structure galaxies. The reason for cross-matching cluster galaxies when we are only interested in those in the field will be discussed in the following section.

\subsection{Checking catalogues values consistency}
\subsubsection{Reasons for checking catalogues values}

\begin{figure}[H]
	\centering
	\begin{minipage}[c]{0.49\linewidth}
		\includegraphics[width=\linewidth]{../Plots/Error_on_inc_versus_b_a.pdf}
		\subcaption{Error on inclination as a function of $b/a$ and its absolute error. Contours of $\Delta (b/a)$ are plotted in black dashed lines with their corresponding value.}
	\end{minipage}
	\hfill
	\begin{minipage}[c]{0.49\linewidth}
		\includegraphics[width=\linewidth]{../Plots/RelError_on_inc_versus_b_a.pdf}
		\subcaption{Error on inclination as a function of $b/a$ and its relative error. Contours of $\epsilon (b/a)$ are plotted in black and red dashed lines with their corresponding value.}
	\end{minipage}
	\caption[Error on inclination as a function of $b/a$ and its error.]{Error on inclination as a function of $b/a$ and its error. Left: as a function of the absolute error on $b/a$ ($\Delta (b/a)$). Right: as a function of the relative error on $b/a$ ($\epsilon ( b/a)$). Red contours correspond to values for which there is a 50\% and 100\% error on $b/a$.}
	\label{fig:erreur_inclinaison}
\end{figure}

As a first step, we must select a sample based on relevant criteria. This is meant to ensure us to have reliable morphological and kinematical parameters and to reduce statistical errors. Given that any kinematical modelling relies on prior morphological information (galaxy centre, ellipticity, $\rm{PA}$), we can only use a combination of values derived from spectral fitting, for instance the Signal to Noise Ratio ($\rm{SNR}$) as described in Eq.\,\ref{eq:SNR}, and morphological modelling such as a measure of a galaxy radius to select our sample.  

Before this internship, spectral fitting on the integrated spectra of the galaxies had already been done, and we combined our data with morphological information from COSMOS catalogues as discussed in the previous section. Potentially useful morphological information included half-light radii, magnitudes, ratios of minor to major axis ($b/a$) or equivalently a measure of the ellipticity of the galaxies. Nevertheless, using this data without checking first how well it compares to values found in other catalogues and/or derived using different softwares/models could lead to high biases and uncontrolled errors. Thus, before discussing any selection criteria for our sample, we must first assess the reliability of the parameters we are going to use in later sections. \\

Important values to check are the half-light radius, as it will be used to select our sample, the $b/a$ ratio and the $\rm{PA}$ since these are prior information for the kinematical modelling. We also checked that there was a correlation between GALFIT and the catalogues magnitudes. The axes ratio has a crucial importance since it is directly related to the inclination of the galaxy on the sky through Eq.\,\ref{eq:inclinaison}. Given a certain error $\Delta (b/a)$, and using the usual formula for computing the error $\Delta f = | \partial_x f | \Delta x$ of a function $f$, we find for the inclination

\begin{equation}
	\Delta i = \Delta (b/a) \left | \frac{b}{a} \left ( 2 - \frac{b}{a} \right ) \right | ^{-1/2}
\end{equation}

This is illustrated in Fig.\,\ref{fig:erreur_inclinaison} where $\Delta i$ has been plotted as a function of $b/a$ and its error (absolute on the left, relative on the right). Contours of the error on $b/a$ have been over-plotted to show how evolves $\Delta i$ given a fixed error on $b/a$. As expected, the higher the error on $b/a$ the higher the error on $i$. An error as high as 50\% could yield $\Delta i \approx \SI{27}{\degree}$, though this value is reached for $b/a \approx 1$ where the axes ratio is the least constrained by the morphology. A more appropriate error on $b/a$ of 20\% gives a maximum $\Delta i$ slightly above $\SI{10}{\degree}$, which is correct. 

Since the inclination affects the galaxy maximum rotational velocity, and so potentially on the classification of galaxies as rotationnaly supported or dispersion dominated (see Section insert ref here), this indicates us that for any proper kinematical modelling we must check carefully that the values of axes ratios are consistent between catalogues.

\subsubsection{Catalogues used for comparison}

As stated in previous sections, we cross-matched our catalogue of galaxies detected by MUSE in COSMOS with Cassata's and Tasca's, two tables with morphological information for the galaxies in \shortciteA{laigle_cosmos2015_2016}. 

However, as can be seen in Fig.\,\ref{fig:comp_radii} and \ref{fig:comp_radii_with_bulge_or_disk_radius}, we found large discrepancies between the parameters. Thus, to better understand their origin, we chose to cross-match our catalogue with another one (also based on \shortciteA{laigle_cosmos2015_2016}) from Zurich. This table has fewer HST counterparts of MUSE galaxies than in the other two but it contains additional morphological information which we can use for the comparison. 

In addition to that, we already had GALFIT morphological information on $\sim 500$ group galaxies with strong confidence in their value. Therefore, we chose to compare the data in the three morphological catalogues based on \shortciteA{laigle_cosmos2015_2016} with that of GALFIT.

\subsubsection{Total magnitudes}

\begin{figure}[htbp]
	\includegraphics[width=\linewidth]{../Plots/catalogMag_against_GalfitMag_corrected.pdf}
	\caption[Comparison between magnitudes]{Comparison between the morphological catalogues magnitudes and that of GALFIT for cluster galaxies. Magnitudes from the catalogues agree well between each other. Left: compared with GALFIT disk magnitude only. The slope is too high and a few points are scattered far from the line. Right: compared with the total GALFIT magnitude as defined in Eq.\,\ref{eq:tot_mag_final_version}. We find a good linear relation with \textcolor{red}{poor scatter}.}
	\label{fig:comp_mags}
\end{figure}

The first value we can easily compare is the total magnitude. Cassata's, Tasca's and Zurich's catalogues provide a measure of the total magnitude derived from fitting with SExtractor a single Sérsic profile with a free Sérsic index $n$ on HST images.

Given that GALFIT light profile was modelled using two Sérsic profiles with fixed Sérsic indices ($n = 1, 4$), we had two measures of the magnitude of these galaxies: one for the bulge component $m_{\rm{b}}^{\rm{GF}}$, and another for the disk component $m_{\rm{d}}^{\rm{GF}}$. To have a meaningful comparison between magnitudes, we need to compute the GALFIT total magnitude by combining the bulge and the disk components. Both are defined as

\begin{equation}
	m_i^{GF} = -2.5 \log_{10} \left ( F_i^{\rm{GF}} \right ) + \rm{C}
	\label{eq:disk_bulge_lum}
\end{equation}
 where $i = \rm{b, d}$ represents either the bulge or the disk, $F = L/{4 \pi D^2}$ is the flux of the galaxy in some band, $L$ its intrinsic luminosity, $D$ its cosmological luminosity distance to us and $\rm{C}$ a constant depending on the band used.

Considering that the two components have different luminosities but are located at the same distance, we can add the fluxes together. Thus the total GALFIT magnitude can also be written as

\begin{equation}
	m_{\rm{tot}}^{\rm{GF}} = - 2.5 \log_{10} \left ( F_{\rm{b}}^{\rm{GF}}  + F_{\rm{d}}^{\rm{GF}} \right ) + \rm{C}
	\label{eq:tot_mag}
\end{equation}

Inverting Eq.\,\ref{eq:disk_bulge_lum} to get the components flux as a function of their magnitude and inserting it into Eq.\,\ref{eq:tot_mag} yields

\begin{equation}
	m_{\rm{tot}}^{\rm{GF}} = -2.5 \log_{10} \left [ 10^{-\frac{m_{\rm{b}}}{2.5}} + 10^{-\frac{\rm{m_d}}{2.5}} \right ]
	\label{eq:tot_mag_final_version}
\end{equation}

This is the value that should be compared with the three catalogues magnitudes. Fig.\,\ref{fig:comp_mags} shows how these scale with each other and with GALFIT disk magnitude on the left, and the total magnitude from Eq.\,\ref{eq:tot_mag_final_version} on the right. As expected, the catalogues give the same value except for a few points. We see that the total GALFIT magnitude gives a much better, poorly scattered linear relation with the catalogues magnitudes. Even though there is an offset between GALFIT and the catalogues, this is due to using different conventions for the constant term in Eq.\,\ref{eq:disk_bulge_lum}.

The same comparison was done on field galaxies, except we did not have GALFIT magnitudes in this case. We also found a good agreement between the catalogues magnitudes.

\subsubsection{Morphological type classification}
\label{subsubsec:classification}

\begin{figure}[htbp]
	\centering
	\includegraphics[width=\linewidth]{{../Plots/comparisonClassTypes}.pdf}
	\caption[Morphological types comparison]{Comparison between morphological types given in Tasca and Zurich catalogues against that of Cassata. The three classifications of Tasca are those described in Section\,\ref{subsubsec:classification}. Galaxies are labelled as follows: E for ellipticals, Sp for spirals/disks-like, Irr for irregulars. The percentage of galaxies falling into the given classes is indicated in red and the method compared with Cassata's is shown on the top left corner of each plot. We find good agreement between Tasca and Cassata types but not between Cassata and Zurich.}
	\label{fig:morpho_comp}
\end{figure}

We might expect to have some discrepancies in our data because of the models used between GALFIT and SExtractor/GIM2D. A way to check this effect is to study how these differences scale with the morphological type of the galaxies. For instance, if we use the disk half-light radius of GALFIT to compare with that of SExtractor, we might expect to have some scatter in our relation for the elliptical galaxies as the disk component is not the best one to describe them.\\

To see how these relations scale with morphological types, we can use the classification given in the three morphological catalogues. These classifications are based on methods which can be quite different and which can use morphological parameters in different ways. A more detailed explanation of the parameters used, of how these methods work, of their strength and weaknesses can be found in Appendix \ref{appendix:classification}. We provide below a short introduction to these classifications:

\begin{itemize}
	\item Cassata's catalogue gives a classification based on morphological parameters they derived with SExtractor. To do so, they use a reference of $500$ galaxies with known parameters which they visually classify as either elliptical, disk-like/spiral or irregular. From this set, each time a new galaxy must be classified, its $11$ closest neighbours are inspected and the most frequent class is assigned to the galaxy.
	
	\item Tasca's catalogue gives different classifications based on three methods. The first one is similar to the one used by Cassata. This is also the classification they recommend to use because this is the one they put the more their trust in. The second one uses the technique described in (insert Abraham 1996 here) using the asymmetry and concentration parameters. The last one uses a support vector machine to classify galaxies.
	
	\item Zurich's catalogue gives a single classification called Zurich Estimator of Structural Type (ZEST) which is described in \shortcite{Scarlata2007}. This method is based on a Principal Component Analysis (PCA). They decided to keep the first three Principal Components (PA) which retain most of the information present in the original five parameters (concentration, asymmetry, Gini coefficient, second-order moment of the brightest pixels producing $20\%$ of the total flux and the ellipticity of the galaxy).
\end{itemize}

The morphological types given in Tasca's and Zurich's catalogues are compared against the class given by Cassata in Fig.\,\ref{fig:morpho_comp}. We observe a good agreement between Cassata's and Tasca's types with just a few elliptical galaxies labelled as disk-like and vice versa. Roughly $50\%$ of the galaxies appear to be elliptical. On the contrary, Zurich's classification seems to label more than $70\%$ of the galaxies as disk-like, including a large number of elliptical galaxies .\\

Considering the recommendation of Tasca to use its Int class and since we find a good agreement between Cassata's morphological type and those given by Tasca, we decided to use and to stick to Cassata's class throughout this work whenever we needed to separate galaxies between elliptical/disk-like/irregulars. This choice also ensured us to have the largest sample possible with a coherent classification as Cassata's catalogue is the one with the largest number of HST counterparts of MUSE galaxies in the COSMOS field. Because of the incompatible results between Zurich's and Cassata's/Tasca's types, we decided to put aside these values and not use them, though this shall require further investigation in future work to assess the origin of these discrepancies.

\subsubsection{Half-light radii}
\label{sec:comp_radii}

\begin{figure}[htbp]
	\centering
	\includegraphics[width=\linewidth]{{../Plots/plotsWithColourCoding/relErr_against_GalFit1.5LightRadius_colourCoded_CassataType}.pdf}
	\caption[Comparison between half-light radii]{Comparison between half-light radii from the morphological catalogues and the radius of GALFIT disk component. The relative error is shown for the three catalogues (from top to bottom, Zurich, Cassata and Tasca). Points have been colour coded according to their classification given in Cassata's catalogue (Irr for irregular, Sp for spiral/disk, E for ellipticals). Left: the full range is plotted. Right: a zoom on the points with $R_{1/2}^{\rm{GF}} \geq \SI{5}{px}$. The last plot on the bottom combines the information within the three plots on the left.}
	\label{fig:comp_radii}
\end{figure}

\begin{figure}[htbp]
	\centering
	\includegraphics[width=\linewidth]{{../Plots/plotsWithColourCoding/relErr_against_galFit1.5LightRadius_colourCoded_matchAllTypes_TwoGalFitRadii}.pdf}
	\caption[Comparison between radii against disk/bulge radius]{Comparison between catalogues radii and that of GALFIT. This figure is similar to Fig.\,\ref{fig:comp_radii} but we use GALFIT bulb radius for elliptical galaxies. The catalogues appear to overestimate the elliptical galaxies radii when using GALFIT bulge radius.}
	\label{fig:comp_radii_with_bulge_or_disk_radius}
\end{figure}

One of the most important parameters we have to check before the selection is the half-light radius of our galaxies. Indeed, if we underestimate it, we might remove from our sample resolved galaxies and therefore reduce our statistics. On the other hand, overestimating it would give us too many unresolved galaxies for which we would spend time performing the cleaning routine without being able to perform their kinematical analysis in the end.
 
Hence, it is mandatory to thoroughly check the values of the half-light radius from the three catalogues against that of GALFIT, and understand the origin of any discrepancies if there happens to be some. \\

We found a quite large disagreement between our GALFIT half-light radius and those given in the morphological catalogues, as well as among them. This is illustrated in Fig.\,\ref{fig:comp_radii} where the half-light radii in the catalogues are compared against that of GALFIT. Galaxies are colour coded according to the classification given in Cassata's catalogue. We checked that using Tasca's classifications as described in Sec.\,\ref{subsubsec:classification} did not change our conclusions. In these plots, we decided to use for the x-axis the half-light radius of the GALFIT disk component for all the galaxies, even though we might expect the ellipticals to be better described by their GALFIT bulge half-light radius.

The catalogues radii seem to be overestimated with respect to that of GALFIT for low $R_{1/2 , \rm{d}}^{\rm{GF}}$. This was expected as catalogues half-light radii are computed using SExtractor. Since it does not take into account the PSF in its fitting routine, we expect the PSF to dominate more for galaxies with small angular sizes. On the contrary, GALFIT does take into account the PSF in its calculations and therefore we expect the half-light radius of GALFIT to be smaller than SExtractor's when reaching low values of $R_{1/2 , \rm{d}}^{\rm{GF}}$. 

When focussing on galaxies with a GALFIT radius larger than the HST-ACS PSF $\rm{FWHM}$ which is around $\sim \SI{0.15}{"}$ ($4 - \SI{5}{px}$), we observe a global underestimation for all the catalogues, up to roughly 50\%. This scatter is mainly due to elliptical galaxies. On the contrary, radii of disk-like galaxies have the least scatter and biais, especially the values given in Zurich's catalogue. This different behaviour between elliptical and disk-like galaxies might be explained, as mentioned above, by the fact we are using the half-light radius of GALFIT disk component to asses the reliability of the elliptical galaxies half-light radii from the catalogues. This is probably not the best choice, and a better may be to use the bulb components, which better describes the light profile of an elliptical galaxy, and its half-light radius to study elliptical galaxies.\\

If we decide to split the galaxies into two categories, ellipticals and disks/irregulars, and if we use for the first category $R_{1/2 , b}^{\rm{GF}}$, and for the second $R_{1/2 , d}^{\rm{GF}}$ we find that elliptical galaxies half-light radii in the catalogues are now overestimated as shown in Fig.\,\ref{fig:comp_radii_with_bulge_or_disk_radius}. This result, and the underestimated values when using the disk radius for the elliptical galaxies, indicates us that elliptical galaxies seem to be neither dominated (in terms of radius) by the disk component, nor by the bulge in the GALFIT model. It is thus necessary to directly compute an overall half-light radius by integrating the galaxies light profile given in Eq.\,\ref{eq:GALFIT_light_profile} \\


CONCLURE SUR QUEL PARAMETRE UTILISER ET QUELLE CORRECTION APPORTER.























\subsection{SNR and size selection}
\label{sec:cut}

\subsubsection{Size selection}
\label{sec:cut_size}

\begin{figure}[t]
	\centering
	\includegraphics[width=\linewidth]{../Plots/FWHM_variation_with_lambda.pdf}
	\caption[PSF FWHM variation with wavelength.]{PSF FWHM variation with wavelength for the 13 FoVs as measured by Valentina. At least two values of the FWHM were derived from stars in the FoVs by fitting a Moffat distribution to their light profile. We assumed a linear evolution with wavelength. Strong fluctuations appear depending on the observed FoV.}
	\label{fig:FWHM_var_lambda}
\end{figure}

Since we are interested in keeping well resolved field galaxies, we need to apply relevant criteria in order to select the right galaxies. The most obvious parameter we can use to make our selection is the size of the galaxy. We already checked  in Section\,\ref{sec:comp_radii} biases which might arise and their origins, and we corrected them whenever possible. Thus, we only need to define a size criterion to select our sample. \\

Following the earlier work done in \shortciteA{Bacon2015} and \shortciteA{Bacon2017}, the MUSE Point Spread Function (PSF), that is the pattern we obtain when we observe a point-like source with MUSE, is most well described by a \shortciteA{MoffatProfile} profile

\begin{equation}
	I_{\rm{PSF}}(r) = I_0 (1 + (r/\alpha)^2 ) ^{- \beta}
	\label{eq:PSF}
\end{equation}
where $r$ is the radial distance to the centre and $\alpha$, $\beta$ are two seeing dependant parameters. In our case, we are interested in the Full Width at Half Maximum (FWHM) since it is directly related to the seeing conditions and since it gives us information about the minimum spatial extent within which we start to loose information.
The FWHM can be easily derived from the equality $I_{\rm{PSF}} ( \rm{FWHM}/2) = I_0 /2$ using Eq.\,\ref{eq:PSF}, from which we get the following relation

\begin{equation}
	\rm{FWHM} = 2 \alpha \sqrt{2^{1/\beta} - 1 }
	\label{eq:FWHM}
\end{equation}

According to the aforementioned articles the value of $\beta$ is expected to remain roughly constant and, additionally, we would expect from differential image motion theory (insert this paper here when read 10.1086/342683) the FWHM to linearly decrease with wavelength. 

All galaxies are observed via their [OII] doublet at the same rest-frame wavelength. But, given that they are all field galaxies located at a different redshift $z$, we actually observe them at wavelengths covering the entire MUSE spectrum, that is we have the usual relation

\begin{equation}
	\lambda_{\rm{obs}} = \lambda_{\rm{em}} ( 1 + z )
\end{equation}
where $\lambda_{\rm{em}}$, $\lambda_{\rm{obs}}$ are respectively the emitted (rest-frame) and observed wavelengths. Therefore, there is not only one $\rm{FWHM}$ value per field (the $\rm{FWHM}$ will also vary with MUSE fields as seeing conditions change from date to date), but one per galaxy, which we can derive if we know the two parameters (slope and offset) which characterise the linear evolution of the $\rm{FWHM}$ with wavelength. To derive this relation, we need at least two measures of the $\rm{FWHM}$ at two different wavelengths per field. Assuming seeing conditions are similar within a field (no spatial dependence), we can use the same linear relation per MUSE field to compute the $\rm{FWHM}$ for the [OII] wavelength at the redshift of the galaxy. \\

As discussed above, $\beta$ should remain constant, which means the $\rm{FWHM}$ is entirely described by the parameter $\alpha$ through Eq.\,\ref{eq:FWHM}. The measure of $\alpha$ had already been done by V. Abril-Melgarejo on at least two stars per field. For each star, two measures of $\alpha$ were computed by fitting a Moffat profile onto their [OII] and [OIII] images at the redshift of the observed structure (see Table \ref{table:MUSEfieldsProp}).


Though a more rigorous modelling of the wavelength variation of the PSF $\rm{FWHM}$ including both more data points and potentially higher order terms is mandatory for future analysis, we decided to stick to this values in the present work, keeping in mind the large uncertainties which will affect the velocity dispersion maps in the modelling section. A representation of the $\rm{FWHM}$ variation with wavelength for 15 out of 16 observations is shown in Fig.\,\ref{fig:FWHM_var_lambda}. We only have missing values of the $\rm{FWHM}$ for CGr114. Most MUSE fields have $\rm{FWHM}$ values below $\SI{0.7}{"}$ which is not surprising given that it was one of the constraints of the observations. CGr23 is the only FoV to have its $\rm{FWHM}$ above $\SI{0.7}{"}$ for almost every wavelength. Only galaxies further than $z \approx 1.11$. In Table.\,\ref{table:MUSEfieldsProp}, the seeing is the average over the OBs for the [OII] doublet at the redshift of the group which is around $1.17$. Hence the value below $\SI{0.7}{"}$. \\

Considering that the $\rm{FWHM}$ is a measure of how spread a point-like source is, and since we are interested in working with resolved enough galaxies in order to better constrain their kinematics, we would like galaxies to have a characteristic size at least above the $\rm{FWHM}$. Given that for almost all the fields we have an upper limit on the $\rm{FWHM}$ of about $\SI{0.7}{"}$, we decided to use this upper limit as our size selection criterion

\begin{equation}
	R_{1/2} \geq \SI{0.35}{"} \approx \rm{FWHM}/2
\end{equation}

Moreover, according to \shortciteA{Swinbank2017} who compared the half-light radius of the nebular [OII] emission in MUSE images with that of their HST counterpart in ACS \textit{I} or WFC \textit{H}-band, the [OII] half-light radius seems to scale with the HST radius as 

\begin{equation}
	R_{1/2}^{\rm{OII}} = (1.18 \pm 0.03) R_{1/2}^{\rm{HST}}
\end{equation}

Thus, we expect the galaxies in the MUSE images to be larger than their HST counterparts, which means our choice of lower limit for the morphological radius in our sample should eliminate most of the unresolved galaxies. Though, we also risk to eliminate resolved galaxies in our selection.

\subsubsection{SNR selection}
\label{sec:cut_SNR}

The other information we can use to select our sample is the Signal to Noise Ratio ($\rm{SNR}$), which tells us how well our galaxy is detached from the background. The $\rm{SNR}$ is generally derived as the ratio between the source's signal and the background level. The noisier an image, the lower the SNR is.

As explained in later sections, the galaxies must be automatically and then manually cleaned before fitting a kinematical model on the velocity maps in order to remove any pixel dominated by noise which might compromise the fit. One of the criteria used by the routine to decide whether a pixel belongs to the galaxy is a lower limit on the $\rm{SNR}$ of pixel (typically $5$). Thus, if we want to have enough detection in our cleaned maps to perform the kinematical modelling, we must select galaxies with a strong enough SNR.

In our case, \textsc{platefit}, described in \shortciteA{}, was run on the integrated spectrum of the galaxies in order to derive their spectral features. From the parameters returned by CAMEL, we used the [OII] flux and its error to derive the SNR as

\begin{equation}
	\rm{SNR} = \frac{\rm{[OII] \,\, flux}}{\rm{[OII] \,\, flux \,\, error}}
\end{equation}

This is the [OII] $\rm{SNR}$ from the integrated galaxy spectrum computed using pixels within data cubes restricted around the galaxy. Since the typical SNR value used by the routine to clean the maps is around $5$, we decided as a first step to choose an SNR lower limit of $10$, allowing us to keep galaxies with strong enough detection after the automatic cleaning is performed.

\subsection{Characterisation of the sample}

\subsubsection{Selecting galaxies}

\begin{figure}
	\centering
	\includegraphics[width=\linewidth]{{../Plots/Selection_plots/SNR_vs_R_halfLight_arcsec}.pdf}
	\caption[Selected sample]{Full sample of $261$ field galaxies with our selection box over-plotted (hatched area). Galaxies with $5 \leq \rm{SNR} \leq 15$ and $0.25" \leq R_{1/2} \leq 0.45"$ (bounds plotted as dashed lines) were visually inspected to check how many resolved galaxies we might have loosed. Selection criteria from Sec.\,\ref{sec:cut} are also plotted as solid red lines. No specific trend appear with redshift.}
	\label{fig:sample_selection}
\end{figure}

We decided to visually inspect galaxies around the $\rm{SNR}$ and the size cuts to quantify
\begin{enumerate*}[label={(\alph*)}]
	\item how many resolved galaxies would be lost if we applied the criteria given in Sec.\,\ref{sec:cut_size} and \ref{sec:cut_SNR} (false negative)
	\item how many unresolved galaxies would be selected in our sample (false positive).
\end{enumerate*}
To do so, we defined four boxes: $5 \leq \rm{SNR} \leq 10$, $10 \le \rm{SNR} \leq 15$, $0.25" \leq R_{1/2} \leq 0.35"$ and $0.35" \le R_{1/2} \leq 0.45"$ containing respectively $46$, $20$, $58$ and $49$ galaxies. We then ran the automatic cleaning routine on all the galaxies in these boxes. After visually inspecting their size in the cleaned maps, we classified them as resolved or not. These boxes and the $\rm{SNR}$, half-light radius and redshift of the full sample of field galaxies before selection are shown in Fig.\ref{fig:sample_selection}. \\

Among the $261$ field galaxies, $103$ fall into the limits imposed in Sec.\,\ref{sec:cut}. We find that roughly $26\%$ of the galaxies below the cuts are actually false negatives and that $15\%$ above them are false positives. Since this classification is purely visual, slightly relaxing the constraints on when a galaxy is resolved or not allow these values to vary by roughly $10\%$. Hence, it seems our choice of lower bounds for the $\rm{SNR}$ and the half-light radius is fine, though in future works we might increase our sample by a factor of $\sim 1.5$.

\subsubsection{Redshift distribution}

\begin{wrapfigure}{r}{.5\linewidth}
	\centering
	\includegraphics[width=\linewidth]{{../Plots/Selection_plots/cumulative_hist_redshift}.pdf}
	\caption[Redshift distribution]{Redshift distribution of the total sample (grey), selected (green) and unselected (blue) galaxies. Redshift bins of size $0.1$ have been used. Top: density plot as a function of redshift. Bottom: cumulative distribution. We lack most of the galaxies at redshift $1.4$. Other redshift bins do not loose to many galaxies with respect to the unselected ones except maybe at $z \approx 0.5 $.}
	\label{fig:redshift_distribution}
\end{wrapfigure}

The choice of the [OII] $\lambda\lambda 3729, \SI{3729}{\angstrom}$ doublet for the kinematical analysis was due in part to the large range of redshift it covered given the MUSE observed wavelengths. We therefore had in our catalogue galaxies spanning a redshift range from $0.3$ to about $1.4$. As show in Fig.\,\ref{fig:redshift_distribution}, after our selection we loose the majority of the most distant galaxies as well as a significant fraction of galaxies with $z$ around $0.3$, $0.5$ and $0.75$. Globally, we are selecting less galaxies per redshift bin than we are putting aside. The average and median redshifts are $0.749$ and $0.719$ for the selected galaxies, $0.809$ and $0.750$ for the unselected ones. Thus, our sample remains consistent in terms of redshift even after performing the $\rm{SNR}$ and size selections.

\subsubsection{Mass-SFR relation}

\begin{figure}[hbtp]
	\centering
	\includegraphics[width=\linewidth]{{../Plots/Selection_plots/SFR_vs_mass_withoutFAST}.pdf}
	\caption[SFR-mass relation]{SFR-mass relation for selected (filled circles) and unselected (open circles) galaxies. Left: SFR extracted from the morphological catalogues. Right: the redshift corrected SFR as given in Eq.\,\ref{eq:SFR_corrected_redshift}. We obtain the usual main sequence of galaxies as a linear relation between SFR and mass (in log-log space), as well as quenched galaxies on the lower right part of the plane which are removed when performing the selection.}
	\label{fig:sfr_vs_mass}
\end{figure}



 


% Kinematical modelling
\clearpage
\section{Kinematical modelling}
\label{sec:kinematical_modelling}

\subsection{Cleaning}

Before trying to fit a model onto the velocity maps to derive the kinematical parameters of our galaxies, we need to remove any pixel which might be dominated by the noise. Such a cleaning is generally done by keeping pixels whose [OII] $\lambda\lambda 3729, \SI{3729}{\angstrom}$ doublet emission line is visually identifiable with respect to the continuum. This quite subjective procedure can be objectified by considering the SNR of each pixel, which is computed in the MUSE pipeline, and by keeping those whose value is above a certain threshold which we chose to be $5$.

However, in some cases, we expect to find a strong continuum emission which we may over-interpret as a spectral feature such as [OII]. Continuum emission comes from various physical processes, most being independent from the others, which implies a larger dispersion in the pixels spectra than in the [OII] case. The dispersion threshold above which we consider pixels are noise dominated will depend on the galaxy redshift. Indeed, considering Gaussian line profiles, we find the spectral dispersion as a function of the LSF $\rm{FWHM}$ of the line by solving $\exp \left \lbrace - ( \rm{FWHM}/2)^2 / (2 \sigma^2 ) \right \rbrace = 1/2$, i.e.

\begin{equation}
	\sigma (z) = \frac{\rm{FWHM}}{2 \sqrt{2 \log 2}}
\end{equation}

We can explicitly express the redshift dependence by defining an equivalent velocity dispersion $\sigma_{\rm{v}}$ as

\begin{equation}
	\frac{\sigma}{\lambda_{\rm{em}} (1+z)} = \frac{\sigma_{\rm{v}}}{c}
\end{equation}

This value defines the usual velocity dispersion we expect to measure from the emission line of a galaxy at redshift $z$. As the Universe expands ($z$ going from $\infty$ to $0$), its bulk motion takes over the local movement of matter, reducing the observed dispersion. We decided to choose a velocity dispersion threshold of $0.8 \sigma_{\rm{v}}$ for all the galaxies throughout this work.

The cleaning was automatically performed with a routine using both thresholds defined above. Cleaned maps were produced and the velocity maps were visually inspected. Isolated pixels were removed even if they had a clearly visible [OII] doublet. Twelve galaxies were unresolved and $3$ were too close to the edges which resulted in missing data. Relaxing the $\rm{SNR}$ threshold to $3$ and removing any pixels on the edges of the galaxies with too weak [OII] emission lines allowed us to recover $6$ galaxies out of $12$. In the case of resolved galaxies, we also inspected their [OII] doublet near the edges and we removed any pixel whose velocity seemed inconsistent with that of its neighbours (a single negative value around positives ones for instance). 

\subsection{Kinematical parametrisation}

The velocity maps we had from the MUSE pipeline only give us information on a fraction of the velocity. Indeed, the velocity of a given pixel is computed using the well known Doppler effect which will redshift pixels going away from us and blueshift those approaching, i.e.

\begin{equation}
	\frac{\lambda_{\rm{obs}} - \lambda_{\rm{em}}}{\lambda_{\rm{em}}} = \frac{v}{c}
\end{equation}
with $v$ positive if it recedes from us and negative if it approaches. Special relativistic effects being negligible in this case, it means the only measure we have is the radial component of the gas velocity within the pixels. Depending on the basis we choose to decompose our velocity vector, this component will be written differently. In our case, we will use the coordinate system defined in \shortciteA{Epinat2008}. Using cylindrical coordinates relative to the galactic plane, we can write the observed projected radial component as 

\begin{equation}
	V_{\rm{obs}} ( R, \theta , i, V_{\rm{sys}} , \rm{PA} ) = V_{\rm{sys}} + \left ( V_{\rm{rot}} (R) \cos \theta + V_{\rm{exp}} (R) \sin \theta \right ) \sin i + V_z (R) \cos i
	\label{eq:v_obs}
\end{equation}
where $V_{\rm{obs}}$, $V_{\rm{sys}}$, $V_{\rm{rot}}$, $V_{\rm{exp}}$ and $V_z$ are the observed,  systematic, rotational, expansion and vertical components of the velocity respectively. In practice, spectroscopic measurements give us $\Delta V_{\rm{obs}} = V_{\rm{obs}} - V_{\rm{sys}}$. In Eq.\,\ref{eq:v_obs}, the dependence on the major axis (kinematical) position angle is implicit as $\theta$ is defined as the angle in the galactic plane starting from the major axis. Thus, $\theta = 0 \degree$ ($90 \degree$) when we measure the velocity of a point on the major (minor) axis. The dependence of $V_{\rm{obs}}$ with the radial distance $R$ to the centre of the galaxy also makes implicit its dependence on the galaxy centre position.

A representation and the definitions of the coordinate system, angles and vector decomposition can be found in Appendix \ref{appendix:coordinates}. Both $R$, $\theta$ and the galaxy centre $(x_{\rm{c}} , y_{\rm{c}})$ can be derived from the images, but the inclination of the galaxy must be know in advance. As it will be explained later, models poorly converge when the inclination is let free to vary. Hence, it must be used as a fixed input. More importantly, assuming the rotation curve along the major axis reaches a maximum rotational velocity or at least a plateau, as should be the case in dark matter dominated galaxies (ref here), we see from Eq.\,\ref{eq:v_obs} that the maximum rotational velocity will scale as

\begin{equation}
	V_{\rm{obs}}^{\rm{max}} (R_{\rm{max}}, i) = V_{\rm{rot}} (R_{\rm{max}}) \sin i
\end{equation}
where $R_{\rm{max}}$ represents the radius at which the maximum velocity or the plateau is reached. Thus, just by looking at the rotation curve and by measuring its maximum value, we cannot raise the degeneracy between the inclination, which will lower the velocity, and the real one . This explains the care taken in checking prior information we had on the $b/a$ ratio from which is computed the galaxy inclination.


\subsection{Model}
\label{subsec:model}

Given the smooth cleaned velocity maps, we can deduce the kinematical parameters of our galaxies by fitting 2D velocity models. Many can be used, but we describe here the one we utilised in this work. 
We used a parametric model relying on the assumption of a thin disk. This is no longer the case when galaxies are seen nearly edge-on ($75 \degree \lesssim i \lesssim 90 \degree$) due to the increasing depth leading to higher values of opacity. The second assumption is that we can actually model our galaxies with a rotation curve and so that we do not have non-rotational bulk motions. In our case, we assume we have smooth velocity fields with no bar, no arms and no clumps such as HII regions. Nevertheless, based on previous morpho-kinematical studies (ref here), we might expect to find a significant fraction of our galaxies to be dispersion dominated, some with highly perturbed morphologies due to merging in some cases.
In our model, $4$ morphological and $1$ kinematical parameters are used. The inclination of the galaxy, its centre right ascension and declination and the $PA$ of the major axis must be given as fixed values. The kinematical parameter corresponds to the systematic velocity, that is the radial component of the velocity due to the expansion of the Universe.



% Data nalysis
	\clearpage
\section{Morpho-kinematics analysis}
\label{sec:morpho_kin_analysis}

\subsection{$V_{\rm{max}} /\sigma_{\rm{v}}$ distribution}

\begin{figure}[hbtp]
	\centering
	\vspace{-5pt}
	\includegraphics[width=\linewidth]{{../Plots/V_sigma}.pdf}
	\caption[$V_{\rm{max}} /\sigma_{\rm{v}}$ distribution]{Left: $V_{\rm{max}} /\sigma_{\rm{v}}$ distribution. Right: $V_{\rm{max}} - \sigma_{\rm{v}}$ diagram with galaxies colour coded according to their redshift and the corresponding distributions in terms of maximum rotation velocity and velocity dispersion (median value represented as a plain black line and the dispersion around the median as black dashed dotted lines). The sample of intermediate-redshift galaxies observed with MUSE in the HUDF (Contini et al. 2019, in preparation) is also plotted for comparison. In each plot, two lines, a red plain one for $V_{\rm{max}} /\sigma_{\rm{v}} = 1$ and a blue dashed one for $V_{\rm{max}} /\sigma_{\rm{v}} = 2$, are over-plotted to show the separation between dispersion dominated systems ($<1$) and rotationally supported galaxies ($>2$). We find that $90\%$ of our sample galaxies are rotationally supported.}
	\label{fig:Vmax_sigma}
\end{figure}

Intermediate and high redshift galaxies are classified in kinematics studies as either dispersion dominated ($V_{\rm{max}} /\sigma_{\rm{v}} < 1$) or rotationally supported ($ > 1 - 2$). The galaxies velocity dispersion is returned by the model as a parameter, as well as the turnover radius $R_{\rm{c}}$, the plateau velocity $V_{\rm{c}}$ and the largest radius where the fit was performed $R_{\rm{last}}$. To remain consistent with other studies, we decided to
\begin{enumerate*}[label=(\alph*)]
	\item compute the maximum rotation velocity of our galaxies at $R_{\rm{max}} = 2.2 R_{\rm{d}}$, with $R_{\rm{d}} \approx 0.5955 R_{1/2}$ the disc scale length (see Appendix \ref{sec:disk_scale_length} for more details).
	\item only keep galaxies with a reliable measure of $V_{\rm{max}}$, that is either if $R_{\rm{max}} < R_{\rm{last}}$ or $R_{\rm{c}} < R_{\rm{last}}$.
	
\end{enumerate*}\\

Our sample is represented in Fig.\,\ref{fig:Vmax_sigma} (coloured filled circles) and compared against the sample of intermediate-redshift galaxies observed with MUSE in the Hubble Ultra Deep Field (HUDF, Contini et al. 2019, in preparation). We find that most galaxies ($90\%$) are rotationally supported ($V_{\rm{max}} /\sigma_{\rm{v}} > 2$). This must be put in contrast with other studies where a higher fraction of dispersion dominated galaxies were found (eg. \shortciteNP{Schreiber2009}, \shortciteNP{Law2009}, \shortciteNP{Gnerucci2011}, \shortciteNP{Epinat2012}, \shortciteNP{Contini2016}). The most probable explanation for such a large fraction is that we use more restrictive selection criteria than in other works. These have removed a substantial fraction of galaxies which are spatially barely resolved with MUSE. However, less restrictive criteria, such as the size threshold in the SNR maps used in Contini et al. (2019), allow one to select small galaxies which are dominated by beam-smearing effects. Therefore, it is not that surprising to find fewer dispersion dominated galaxies in our sample.\\

We do not find any significant evolution with redshift. Visually, it seems that galaxies with higher dispersion are found at higher redshift. But splitting the sample around $\sigma_{\rm{v}} \approx 30 - \SI{40}{km/s}$ into two categories and computing the median value of the redshift for both populations does not yield a high enough redshift difference given the large dispersion.

\begin{wrapfigure}{l}{0.6\linewidth}
	\centering
	\includegraphics[width=\linewidth]{{../Plots/V_sigma_size}.pdf}
	\caption[$V_{\rm{max}} - \sigma_{\rm{v}}$ diagram as a function of apparent size]{Same graph as in Fig.\,\ref{fig:Vmax_sigma} right plot where the galaxies have been colour coded according to their apparent size on the sky. The dispersion dominated galaxies all have small sizes, but at larger radii we do not find a clear link between size and $V_{\rm{max}}$ or $\sigma_{\rm{v}}$.}
	\label{fig:V_sgima_size}
\end{wrapfigure}

We checked the effect of size selection on the $V_{\rm{max}} / \sigma_{\rm{v}}$ distribution as shown in Fig.\,\ref{fig:V_sgima_size}. We do not find a clear relation between the half-light radius and the rotation maximum velocity, nor the velocity dispersion. But, we do observe that all the dispersion dominated galaxies, as well as those classified as in between ($1 < V_{\rm{max}} / \sigma_{\rm{v}} < 2$) are all found to be among the smallest ones. Nevertheless, with less than $10$ dispersion dominated objects, it is difficult to conclude. In principle, one should perform the kinematical modelling for the unresolved, beam-smearing dominated galaxies and compare the $V_{\rm{max}} / \sigma_{\rm{v}}$ distribution of this population with ours to see how the size selection criterion would translate in terms of the proportion of dispersion dominated systems.\\

It is possible to relate the ratio between the maximum rotation velocity and the velocity dispersion to the gas fraction in galaxies. Toomre \shortcite{Toomre1964} provided an estimate of the gravitational stability of rotating gas parcels and showed that the rotation of disks ($f_{\rm{c}} \approx \Omega^2 R$) cannot counter-balance their gravity ($f_{\rm{g}} \approx \pi G \Sigma $) if their length is below $L_{\rm{crit}} = (\pi G \Sigma)/\Omega^2$. This can be compared with Jean's instability which occurs when $L > L_{\rm{J}} \approx \sigma_{\rm{v}}^2 / ( G \Sigma )$. The combination of the effects of gravitational and centrifugal accelerations are summarised in the Toomre parameter

\begin{equation}
		Q =  \frac{\kappa \sigma_{\rm{v}}}{\pi G \Sigma_{\rm{gas}}} \approx \frac{\sigma_{\rm{v}}}{f_{\rm{gas}} V_{\rm{max}}}
\end{equation}
where $\kappa \approx V_{\rm{max}} / R$ is the epicyclic frequency, $\Sigma_{\rm{gas}}$ is the gas mass surface density and $f_{\rm{gas}}$ the fraction of gas in the galaxies. In the case of marginally stable disks ($Q \approx 1$) there should therefore be a correlation between $V_{\rm{max}}$ and $\sigma_{\rm{v}}$, though with some scatter as disks will have slightly different gas fractions (see \shortciteA{Tacconi2018}; \shortciteA{Freundlich2019}).




%\newpage
\subsection{Tully-Fisher Relation}

\subsubsection{Recovering the TFR}

\begin{figure}[htbp]
	\centering
	\includegraphics[width=\linewidth]{{../Plots/TFRz}.pdf}
	\caption[Tully-Fisher Relation]{Tully-Fisher Relation
	\begin{enumerate*}[label=]
		\item Left: as a function of redshift.
		\item Right : as a function of $V_{\rm{max}} /\sigma_{\rm{v}}$ ratio.
	\end{enumerate*}		
	In both plots, galaxies have been separated between rotationnaly supported (filled circle), dispersion dominated (cross) and in between (filled square). Error bars correspond to $1\sigma$ uncertainties. The TFR from the EAGLE simulation \shortcite{Schaye2015} and its errors (plain and dashed lines) are over-plotted for comparison. We also over-plotted (grey points) the results of Contini et al., 2019 (in preparation) for galaxies at similar redshifts observed with MUSE in the HUDF.}
	\label{fig:TFR}
\end{figure}

We recover the Tully-Fisher Relation (TFR) of our sample galaxies by comparing their stellar masses, obtained with SED fitting and given in \shortciteA{laigle_cosmos2015_2016} catalogue, against the maximum (inclination corrected) rotation velocity. In Fig.\,\ref{fig:TFR}, we compare our TFR with the one derived from the EAGLE suite of hydrodynamical simulations \shortcite{Schaye2015} of galaxies in a similar mass range. We also add the results obtained in Contini et al. (2019) discussed in the previous section.\\

We find a similar trend between mass and rotation velocity, but globally with an offset towards larger velocities with respect to both the simulations and the MUSE observations in the HUDF. This is consistent with the $V_{\rm{max}} / \sigma_{\rm{v}}$ distribution, but also with the lack of dispersion dominated galaxies. Such objects are at the origin of the dispersion in the TFR at lower rotation velocities and lower masses in Contini et al. (2019) sample in Fig.\,\ref{fig:TFR}. We can also see this effect in our sample where the galaxies with the lowest $V_{\rm{max}} / \sigma_{\rm{v}}$ ratio are the most scattered towards lower rotation velocities (see right plot of Fig\,\ref{fig:TFR}). In terms of redshift, we do not find any evolution of the TFR zeropoint. 


\subsubsection{A size dependent relation ?}

\begin{wrapfigure}{r}{0.6\linewidth}
	\centering
	\includegraphics[width=\linewidth]{{../Plots/TFR_sizeKPC}.pdf}
	\caption[TFR as a function of galaxies physical size]{Same plot as Fig\,\ref{fig:TFR} (similar legend) but galaxies are colour coded according to their physical size as computed from Eq.\,\ref{eq:TFR_phys_size}. We observe a tendency for larger galaxies to be located further up on the TFR (larger masses and rotation maximum velocity).}
\end{wrapfigure}

There does not seem to be any obvious correlation between the TFR of our sample and the galaxies redshift or the angular size as can be seen in Fig.\,\ref{fig:TFR_sizePX}. However, we do expect to see a correlation between their stellar mass and their size \shortcite{Mowla2018}. Thus, knowing their angular half-light radius (we will note it $\theta_{1/2}$) and their redshift, we derived their physical size (half-light radius in $\si{kpc}$, we note it $R_{1/2}$ from now on)

\begin{equation}
	R_{1/2}(z) = \theta_{1/2} d_A (z)
	\label{eq:TFR_phys_size}
\end{equation}

with $d_A (z) = d_c(z) / (1+z)$ the angular diameter distance and $d_c$ the comoving distance. As expected, smaller galaxies have a lower mass and, given the TFR, also have a lower maximum rotation velocity.

\subsubsection{The effect of inclination}

Our TFR in Fig\,\ref{fig:TFR} contains the whole sample of spatially resolved galaxies. Even though the maximum rotation velocity is corrected of the inclination through Eq.\,\ref{eq:V_obs_max}, it still has an impact on the TFR. The kinematics of edge-on galaxies (typically $i \gtrsim 70\degree$) is less constrained because
\begin{enumerate*}[label=(\alph*)]
	\item of the lack of pixels to fit the model.
	\item the assumption of an infinitely thin disc on which is based the model is not necessarily met any more.
\end{enumerate*}
For face-on galaxies ($i \lesssim 30\degree$), we do generally have plenty of pixels (assuming it is spatially extended enough), but the measured velocity on the line of sight is also smaller, which requires a better spectral resolution to produce detailed velocity field maps. In addition to that, the relative error on the inclination is greater when it approaches $0\degree$ as morphological models can easily converge towards non-zero values of $b/a$ (see left plot of Fig.\,\ref{fig:erreur_inclinaison}). We therefore checked how the inclination might have impacted our previous TFR. In Fig.\,\ref{fig:TFR_inc} we separated our galaxies into two categories: face-on and edge-on galaxies with less constrained kinematics, and those with reasonable values of inclination ($30\degree \leq i \leq 70\degree$). Only rotationnaly supported galaxies with $V_{\rm{max}} / \sigma_{\rm{v}}$ are eliminated with this criterion. This is not very surprising since edge-on galaxies need to have a clear disk-like structure to be classified as such, which is generally linked to larger values of rotation velocity. In addition, we expect to have a significant fraction of galaxies with a maximum rotation velocity above the TFR because of the effect of very low inclinations on the sky.


\subsection{Scaling between $\rm{SFR}$ and velocity dispersion}

\begin{wrapfigure}{l}{0.6\linewidth}
	\includegraphics[width=\linewidth]{{../Plots/SFR_sigma}.pdf}
	\caption[Scaling between SFR and $\sigma_{\rm{v}}$]{Scaling relation between the star formation rate and the velocity dispersion. We find an evolution as in the $\rm{SFR} - V_{\rm{max}}$ diagram, but with more scatter}
	\label{fig:SFR_sigma}
\end{wrapfigure}

Since our galaxies are all located on the main sequence, there exists a relation between the $\rm{SFR}$ and the mass, with more massive galaxies having higher star formation rates. Given the TFR, we also have a scaling between mass and maximum rotation velocity. Therefore, it seems natural to see a link between the $\rm{SFR}$ and the rotation velocity in Fig.\,\ref{fig:SFR_Vmax}. Interestingly enough, we do observe a redshift evolution. This is mainly due to the redshift evolution of the $\rm{SFR}$ on the main sequence (see Fig.\,\ref{fig:sfr_vs_mass}). If we observe such a relation, one may ask if higher a $\rm{SFR}$ leads to higher velocity dispersion as well. In Fig.\,\ref{fig:SFR_sigma}, we show the variation of the $\rm{SFR}$ with the velocity dispersion and with redshift. Qualitatively, we can see that the velocity dispersion seems to increase with $\rm{SFR}$ and with redshift, as discussed before, though the scatter is higher than between the $\rm{SFR}$ and $V_{\rm{max}}$. A similar trend, both in terms of $\rm{SFR} - \sigma_{\rm{v}}$ relation and redshift evolution on this sequence, was recently found by \shortciteA{Hung2019} in cosmological simulations from the FIRE project \shortcite{Hopkins2014}. Moreover, if we interpret the velocity dispersion as the effect of energy injection into the gas, it seems consistent to find larger values for higher $\rm{SFR}$. This is the conclusion reached by \shortciteA{Lehnert2013} who compared data obtained with SINFONI with simulations of gas rich galaxies, and who found that most of the velocity dispersion was driven by star formation.


% Conclusion
\clearpage
\section{Conclusion and perspectives}

The main objective of this five months internship is to perform a morpho-kinematical analysis of intermediate redshift ($0.4 < z < 1.4$) galaxies located in low density environments, with [OII] detection, and observed with MUSE in the COSMOS area. The first part consisted in checking the robustness and consistency of morphological parameters we retrieved from HST catalogues, as well as understanding the origin of any discrepancies. We used two criteria, a bias corrected half-light radius and a SNR threshold, to select a sample of $\sim 100$ galaxies which are spatially resolved in our MUSE data. After exploring our sample in terms of redshift and mass-SFR, we cleaned the velocities maps of pixels with fake detection. We modelled the galaxies velocity fields assuming a thin rotating disc, and using a ramp model which takes into account beam-smearing effects, the LSF and the inclination . Using the maximum rotation velocity $V_{\rm{max}}$ and the velocity dispersion $\sigma_{\rm{v}}$ returned by the model, we provide a first morpho-kinematics analysis of our sample. We find a $V_{\rm{max}} / \sigma_{\rm{v}}$ distribution dominated by rotationnaly supported galaxies. We argue that this is most probably due to selection effects and compare it against the results from another survey and a simulation. We also investigate the Tully-Fisher relation, and do not find any evolution with redshift. Finally, we find a correlation between the $\rm{SFR}$ and the velocity dispersion, and we interpret it as an effect of energy injection into the gas during star-formation.\\

Only four months have passed since the beginning. Though the main goal has been achieved, the data analysis is not over yet. The next (and last) month of my internship will be dedicated to this task. The different results provided here and their interpretation will be further investigated, especially the TFR and the origin of its offset with respect to other studies. In the short term these results will also be compared against those obtained by V. Abril Melgajero on the group and cluster galaxies observed in the same fields. But this is only possible if we carefully take into account the different selection criteria used. Nevertheless, we should be able to explore the effect of the environment on the morphological and kinematical properties of galaxies at intermediate redshift and on low mass galaxies as well. 

In the longer term, that is if this is to be continued into a PhD, there is substantial work to do. In terms of methodology, there is the question of the sample selection. By using less restrictive criteria we could almost double its size, but we need to check in more details how these will affect our conclusions. A larger sample also requires more time for the kinematical modelling. This is also an aspect which could be improved by using more reliable values of the galaxies centres. Another important aspect which has not been covered here is the impact of the kinematical model on our conclusions. Other models exist, some based on mass distributions, which would most certainly yield slightly different values of $V_{\rm{max}}$ and $\sigma_{\rm{v}}$. Therefore, the question is: how much different would they be ? In terms of data analysis, we will have only started to answer a few questions about galaxy evolution through cosmic time at the end of the internship. But there are plenty more, from studying the dark matter content of these galaxies (scaling between dynamical and stellar masses across cosmic time ?), to investigating the angular momentum redistribution between dark matter haloes and baryons, as well as understanding the impact of merging and gravitational interaction between galaxies on their morphology and kinematics. 

%%%%%%%%%%%%%%%%%%%%%%%%%%%%%%%%%%%%%%%%%%%%%%%%%%%%%%%%%%%%%
%BIBLIOGRAPHY
%%%%%%%%%%%%%%%%%%%%%%%%%%%%%%%%%%%%%%%%%%%%%%%%%%%%%%%%%%%%%

\clearpage
\renewcommand*{\thesection}{}\textbf{}

\begin{multicols}{2}
	\def\bibfont{\footnotesize}
	\bibliographystyle{apacite}
	{\footnotesize\bibliography{Bibliography.bib}}
\end{multicols}

%%%%%%%%%%%%%%%%%%%%%%%%%%%%%%%%%%%%%%%%%%%%%%%%%%%%%%%%%%%%%
%APPENDICES
%%%%%%%%%%%%%%%%%%%%%%%%%%%%%%%%%%%%%%%%%%%%%%%%%%%%%%%%%%%%%


\appendix
\renewcommand*{\thesection}{\Alph{section}}\textbf{}

% APPENDIX on morphological classification
%\clearpage
\section{Details about morphological classification}
\label{appendix:classification}

We provide in this section a summary of the morphological parameters used by Cassata, Tasca and Zurich for their morphological classification which we used and compared in the present work.

\subsection{Non-parametric values}

Usual galaxy classification methods rely on the same parameters. Some may use all of them when others only concentrate on a few. We give here a description of the main parameters used for this kind of classification.

\subsubsection{Concentration}

The first parameter is the concentration. Two quite different definitions exist, the former using flux levels and the latter isophotes. The oldest definition of the concentration comes from \shortciteA{Kent1985}. In his paper, he describes it as the log of the ratio between the radii enclosing $80\%$ and $20\%$ of the "total" luminosity (respectively $r_{80}$ and $r_{20}$)

\begin{equation}
	C = 5 \log \frac{r_{80}}{r_{20}}
	\label{eq:concentration_Kent_form}
\end{equation}

The concentration parameter can vary in theory from $1$ to infinity. The parameter $C$ as defined in Eq.\,\ref{eq:concentration_Kent_form} should be interpreted cautiously as many definitions of the "total" luminosity (sometimes also called total flux) are used. Some authors who have performed a morphological modelling might call total luminosity the integrated flux up to infinity of their best fit model. Others might use the flux within some fixed aperture (generally a multiplicative factor of the Petrosian radius $r_{\rm{p}}$), such as $2 \, r_{\rm{p}}$ in \shortciteA{Bershady2000}, or $1.5 \, r_{\rm{p}}$ in \shortciteA{Lotz2004}. \\

Another definition comes from \shortciteA{Abraham1994}. In this case, second order moments of the image are computed in order to find an outer elliptical aperture with an area similar to that of the galaxy up to some isophotal limit (generally $2\sigma$). From this ellipse, the semi-major axis can be obtained. A second semi-major axis $30\%$ smaller is derived and an inner ellipse with the same shape is defined. The concentration parameter is then computed as the ratio between the flux within the outer and the inner apertures

\begin{equation}
	C = \frac{\sum_{(x, y) \in \rm{\mathcal{E}_{in}}} I(x, y)}{\sum_{(x, y) \in \rm{\mathcal{E}_{out}}} I(x, y)}
	\label{eq:concentration_abraham_form}
\end{equation}
where $\rm{\mathcal{E}_{in}}$ and $\rm{\mathcal{E}_{out}}$ are respectively the inner and outer ellipses. This definition therefore yields a concentration parameter $C$ between $0$ and $1$.

\subsubsection{Asymmetry}

The asymmetry parameter defines how symmetric/asymmetric a galaxy is. A related parameter was first introduced by \shortciteA{Schade1995} in order to classify galaxies as being symmetric or not, though its definition differed by using a combination of original and residual maps in its computation.

The most common definition of the asymmetry $A$ comes from \shortciteA{Abraham1996}. The map is rotated by $\SI{180}{\degree}$ and is subtracted to the original map. The asymmetry parameter is then computed as half the ratio between the total integrated intensity amplitude in the self-subtracted map and the total intensity amplitude in the original map. Using an absolute value will add a positive term in $A$ from background signal. To get rid of it, the same calculation is performed for an aperture of the same size with only background signal and this value is subtracted from the previous one, i.e.

\begin{equation}
	A = \frac{\sum_{i , j} \left | I(i,j) - I_{180} (i , j) \right | }{2 \sum_{i, j} \left | I(i,j) \right |} - B_{180}
\end{equation}
where $I_{180}$ designates the rotated map and $B_{180}$ is the average asymmetry of the background. This definition, though commonly used, is sometimes replaced by a similar one where square brackets are used instead of absolute values. This is for instance the case in \shortciteA{Conselice1997}. Nevertheless, both definitions yield values of $A$ between $0$ (completely symmetrical) and $1$ (completely asymmetrical).

The value of $A$ can be quite sensitive to the choice of the centre. Methods for finding the best centre position include using the brightest point in a smoothed map or iterating over $X$ and $Y$ till $A$ is minimised.

\subsubsection{Gini coefficient}



% APPENDIX on PSF variation with wavelength
\clearpage

\section{PSF FWHM variation with wavelength}

\begin{figure}[hbtp]
	\centering
	\includegraphics[width=\linewidth]{../Plots/FWHM_variation_with_lambda.pdf}
	\caption[PSF FWHM variation with wavelength.]{PSF $\rm{FWHM}$ variation with wavelength for the 16 MUSE fields (including \textit{best\_seing} and \textit{deep} observations) measured by V. Abril Melgajero and B. Epinat (LAM, Marseille). At least two values of the $\rm{FWHM}$ were derived from stars in the MUSE fields by fitting a Gaussian profile onto their light profile. We assumed a linear evolution with wavelength. Strong variations appear depending on the observed field.}
	\label{fig:FWHM_var_lambda}
\end{figure}

% APPENDIX on kinematics modelling
\clearpage
\section{Additional information about kinematics}
\label{sec:info_on_kinematics}

\subsection{Velocity decomposition}
\label{sec:vel_decompo}

MUSE velocity maps only give us the projected component of the 3D velocity field on the line of sight. This value is computed from the mean of the best fit Gaussian profile of the observed line, and from the Doppler shift formula

\begin{equation}
	\frac{\lambda_{\rm{obs}} - \lambda_{\rm{em}}}{\lambda_{\rm{em}}} = \frac{v}{c}
\end{equation}
with $v$ positive if it recedes from us and negative if it approaches. This means the only measure we have is the radial component of the gas velocity within the pixels. If we use the same coordinate system as defined in \shortciteA{Epinat2008}, we can write the observed projected radial component $V_{\rm{obs}}$ as 

\begin{equation}
	V_{\rm{obs}} ( (x_c , y_c) , R, \theta , i, V_{\rm{sys}} , \rm{PA} ) = V_{\rm{sys}} + \left [ V_{\rm{rot}} (R) \cos \theta + V_{\rm{exp}} (R) \sin \theta \right ] \sin i + V_z (R) \cos i
	\label{eq:v_obs}
\end{equation}
where $(x_c , y_c)$ is the galaxy centre position, $R$ the radial distance in the galactic plane, $i$ the galaxy inclination and $\rm{PA}$ the kinematical position angle (which might be different from the morphological one). In Eq.\,\ref{eq:v_obs}, the velocity components are
\begin{enumerate*}[label={(\alph*)}]
	\item $V_{\rm{sys}}$ the systematic velocity
	\item $V_{\rm{rot}}$ the rotational velocity tangent to the plane of the galaxy
	\item $V_{\rm{exp}}$ the expansion velocity oriented in the radial direction in the galactic plane
	\item $V_z$ the vertical component normal to the plane.
\end{enumerate*}
The dependence on the kinematical $\rm{PA}$ is implicit since $\theta$ is defined as the angle in the galactic plane counted from the major axis. Thus, $\theta = 0 \degree$ ($90 \degree$) when we measure the velocity of a point on the major (minor) axis. The dependence on the galaxy centre position is also implicit as the radial distance is computed relative to this point. \\

Both $R$ and $\theta$  can be easily derived given $(x_c , y_c)$ and the $\rm{PA}$, but the inclination of the galaxy must be know in advance. Indeed, models tend to poorly converge when the inclination is let free to vary. More importantly, assuming the rotation curve along the major axis reaches a maximum rotational velocity or at least a plateau, we deduce from Eq.\,\ref{eq:v_obs} that the maximum rotational velocity will scale as

\begin{equation}
	V_{\rm{obs}}^{\rm{max}} (R_{\rm{max}}, i) = V_{\rm{rot}} (R_{\rm{max}}) \sin i
\end{equation}
where $R_{\rm{max}}$ represents the radius where the maximum/plateau velocity is reached. Thus, just by looking at the rotation curve and by measuring its maximum value, we cannot raise the degeneracy between the inclination, which will lower the velocity, and the real maximum rotational velocity. Hence, it must be used as a fixed input.

\subsection{Computing the disk scale length}
\label{sec:disk_scale_length}

The model returns the minimum, maximum, average and median values for the dispersion map once corrected of beam smearing effects. We also recover the rotation velocity on the plateau $V_{\rm{c}}$, as well as the radius at which it is reached $R_{\rm{c}}$ and the maximum radius where the fit was performed $R_{\rm{last}}$.\\

We compute the maximum velocity at $2.2 R_{\rm{d}}$, with $R_{\rm{d}}$ the disc scale length. This is defined as the radial distance in a morphological exponential disc model where the intensity drops by a factor of $e$ with respect to the central value. From this definition, we can compute its scaling relation with $R_{1/2}$. From Eq.\,\ref{eq:SersicProfile} and given that $I(R_{\rm{d}}) = I_0 e^{-1} = I_{\rm{e}} e^{- b_1 ( R_{\rm{d}} / R_{\rm{e}} - 1 )}$, we find

\begin{equation}
	R_{\rm{e}} = b_1 R_{\rm{d}}
\end{equation}
where $b_1$ is the solution of the equation $2 \gamma (2, b_1) = \Gamma(2)$. This also translates as

\begin{equation}
	\int_0^{b_1} t e^{-t} dt = 1 - e^{-b_1} ( 1 + b_1) = 1/2
\end{equation}
for which we find a solution by making the substitution $b_1 \rightarrow - y - 1$ and using the Lambert W function

\begin{equation}
	b_1 = - \rm{W} \left ( - \frac{1}{2 e} \right ) - 1 \approx 1.67835
\end{equation}
where we only kept the positive, physically meaningful solution. 

%APPENDIX on  morpho-kinematics plots
\clearpage
\section{Additional plots for the analysis}

\subsection{TFR}

\begin{figure}[htbp]
	\centering
	\includegraphics[width=\linewidth]{{../Plots/TFR_sizePX}.pdf}
	\caption[TFR as a function of angular size]{Same plot as in Fig.\,\ref{fig:TFR} but colour coded according to the apparent angular size (half-light radius) on the sky. The legend is similar to that in Fig.\,\ref{fig:TFR}. As for redshift, we do not find a significant evolution, though all the dispersion dominated galaxies with low $V_{\rm{max}}$ are also found to have the lowest sizes, which is in agreement with the results shown in Fig.\,\ref{fig:V_sgima_size}.}
	\label{fig:TFR_sizePX}
\end{figure}



\newpage
\subsection{Effect of the inclination on the TFR}
\begin{figure}[htbp]
	\centering
	\includegraphics[width=\linewidth]{{../Plots/TFR_vel_corrected_of_i}.pdf}
	\caption[Effect of the inclination on the TFR]{Tully-Fisher Relation for galaxies with a reasonable inclination $30\degree \leq i \leq 70\degree$ (filled circles with error bars) compared against face-on ($i<30\degree$) and edge-on ($i>70\degree$) galaxies.}
	\label{fig:TFR_inc}
\end{figure}



\newpage
\subsection{$\rm{SFR} - V_{\rm{max}}$ relation}
\begin{figure}[htbp]
	\centering
	\includegraphics[width=\linewidth]{{../Plots/SFR_Vmax}.pdf}
	\caption[$\rm{SFR} - V_{\rm{max}}$ relation]{$\log_{10} \rm{SFR} - \log_{10} V_{\rm{max}}$ diagram with galaxies colour coded according to their redshift. We find a link between the $\rm{SFR}$ and the maximum rotation velocity. This is consistent with the galaxies main-sequence shown in Fig.\,\ref{fig:sfr_vs_mass} and with the TFR in Fig.\,\ref{fig:TFR}. }
	\label{fig:SFR_Vmax}
\end{figure}

%APPENDIX on morpho-kinematics parameters
\clearpage
\section{Morpho-kinematics parameters of the sample}

\begin{table}[htbp]

	\centering

	\begin{tabular}{c c c c c c c c c c c}

\hline
Galaxy\textsuperscript{1} & $z$\textsuperscript{2} & $\log_{10} \rm{SFR}$\textsuperscript{3} & $\log_{10} M_{*}$ \textsuperscript{4} & $R_{1/2}$\textsuperscript{5} & $R_{\rm{c}}$\textsuperscript{6} & $V_{\rm{c}}$\textsuperscript{7} & $\sigma_{\rm{v}}$\textsuperscript{8} & $\chi^2$ \textsuperscript{9} \\
ID & & ($\si{M_{\odot} \, yr^{-1}}$) & ($\si{M_{\odot}}$) & ($\si{"}$) & ($\si{px}$) & ($\si{km/s}$) & ($\si{km/s}$) & \\
\hline

\texttt{114\_21}        & $0.345$ & $+0.0$ & $8.16 $ & $7.65 $ & $5.1 $ & $59.9 $ & $13.5$ & $0.68$ \\
\texttt{114\_79}        & $0.423$ & $+0.5$ & $9.31 $ & $4.95 $ & $6.5 $ & $111.2$ & $21.0$ & $1.37$ \\
\texttt{114\_99}        & $0.423$ & $-0.1$ & $9.23 $ & $4.97 $ & $4.1 $ & $108.4$ & $20.8$ & $2.74$ \\
\texttt{114\_104}       & $0.539$ & $-0.3$ & $9.45 $ & $1.79 $ & $6.9 $ & $116.2$ & $21.4$ & $0.63$ \\
\texttt{114\_46}        & $0.924$ & $-0.5$ & $8.44 $ & $1.78 $ & $1.0 $ & $84.7 $ & $25.5$ & $0.73$ \\
\texttt{23\_13}         & $0.677$ & $+1.3$ & $9.5  $ & $1.81 $ & $1.0 $ & $54.9 $ & $48.3$ & $2.41$ \\
\texttt{23\_19}         & $0.953$ & $+1.1$ & $9.01 $ & $2.57 $ & $1.0 $ & $15.0 $ & $26.4$ & $0.59$ \\
\texttt{23\_25}         & $0.677$ & $+1.0$ & $10.04$ & $3.48 $ & $8.8 $ & $217.4$ & $42.5$ & $2.14$ \\
\texttt{23\_30}         & $0.676$ & $+1.1$ & $9.87 $ & $3.15 $ & $1.0 $ & $188.8$ & $29.5$ & $4.22$ \\
\texttt{23\_90}         & $0.379$ & $-0.8$ & $8.52 $ & $4.07 $ & $6.6 $ & $79.6 $ & $6.3 $ & $0.44$ \\
\texttt{23\_94}         & $0.941$ & $-0.6$ & $8.97 $ & $2.04 $ & $4.1 $ & $262.2$ & $4.5 $ & $0.85$ \\
\texttt{23\_120}        & $1.159$ & $+1.2$ & $9.58 $ & $3.51 $ & $1.0 $ & $156.6$ & $27.5$ & $4.40$ \\
\texttt{23\_110}        & $0.731$ & $+0.2$ & $9.17 $ & $2.43 $ & $1.0 $ & $113.7$ & $17.7$ & $0.86$ \\
\texttt{26\_59}         & $0.826$ & $+0.1$ & $9.16 $ & $2.37 $ & $8.0 $ & $169.8$ & $24.6$ & $0.60$ \\
\texttt{26\_87}         & $1.333$ & $+1.7$ & $10.03$ & $4.17 $ & $5.1 $ & $308.2$ & $20.8$ & $15.03$ \\
\texttt{26\_136}        & $1.188$ & $+0.0$ & $10.05$ & $2.39 $ & $1.8 $ & $164.3$ & $31.4$ & $0.99$ \\
\texttt{26\_5}          & $0.533$ & $+0.6$ & $11.11$ & $4.75 $ & $1.0 $ & $300.2$ & $51.6$ & $3.66$ \\
\texttt{26\_31}         & $0.562$ & $-0.5$ & $8.37 $ & $4.93 $ & $1.0 $ & $117.5$ & $10.9$ & $1.20$ \\
\texttt{26\_41}         & $0.531$ & $+0.4$ & $9.36 $ & $1.81 $ & $9.0 $ & $315.0$ & $44.7$ & $0.59$ \\
\texttt{26\_71}         & $0.967$ & $+1.0$ & $9.64 $ & $3.45 $ & $2.1 $ & $166.4$ & $30.8$ & $1.94$ \\
\texttt{26\_102}        & $0.554$ & $+1.1$ & $9.91 $ & $3.12 $ & $1.0 $ & $182.0$ & $20.3$ & $3.83$ \\
\texttt{26\_128}        & $0.425$ & $-0.6$ & $8.44 $ & $3.56 $ & $3.1 $ & $98.1 $ & $20.2$ & $0.78$ \\
\texttt{28\_4}          & $0.718$ & $-0.1$ & $9.36 $ & $4.34 $ & $3.9 $ & $106.2$ & $37.8$ & $1.44$ \\
\texttt{28\_5}          & $0.874$ & $+0.5$ & $9.22 $ & $3.09 $ & $2.5 $ & $114.5$ & $55.0$ & $1.85$ \\
\texttt{28\_15}         & $1.181$ & $+2.1$ & $9.7  $ & $2.75 $ & $1.0 $ & $239.7$ & $81.6$ & $4.64$ \\
\texttt{28\_37}         & $0.894$ & $+0.9$ & $9.88 $ & $2.5  $ & $1.0 $ & $230.2$ & $22.8$ & $2.39$ \\
\texttt{28\_44}         & $0.363$ & $-0.6$ & $9.3  $ & $2.28 $ & $1.0 $ & $63.7 $ & $46.9$ & $1.44$ \\
\texttt{28\_55}         & $0.894$ & $+0.4$ & $9.29 $ & $1.91 $ & $2.9 $ & $226.9$ & $35.7$ & $0.94$ \\
\texttt{28\_59}         & $0.737$ & $+0.5$ & $9.51 $ & $1.92 $ & $1.0 $ & $123.2$ & $16.1$ & $0.75$ \\
\texttt{28\_83}         & $0.719$ & $-0.4$ & $8.31 $ & $2.05 $ & $7.2 $ & $54.0 $ & $24.5$ & $1.15$ \\
\texttt{30\_d\_45}      & $1.211$ & $+0.9$ & $9.48 $ & $2.4  $ & $6.3 $ & $163.0$ & $48.1$ & $2.42$ \\
\texttt{30\_d\_67}      & $0.381$ & $+0.8$ & $9.46 $ & $5.6  $ & $5.2 $ & $142.0$ & $38.5$ & $1.49$ \\


	\end{tabular}
	\caption[Morpho-kinematical parameters I]{Table of morphological and kinematical parameters for the sample of selected field galaxies. 1. Galaxy ID with first the COSMOS group number and then the galaxy number, 2. redshift, 3. Star Formation Rate, 4. Stellar mass, 5. Half-light radius, 6. Turnover radius (MUSE pixel), 7. Plateau velocity, 8. Velocity dispersion, 9. Reduced $\chi^2$.}
\label{table:params1}
\end{table}



\newpage
\begin{table}[htbp]

	\centering

	\begin{tabular}{c c c c c c c c c c c}

\hline
Galaxy\textsuperscript{1} & $z$\textsuperscript{2} & $\log_{10} \rm{SFR}$\textsuperscript{3} & $\log_{10} M_{*}$ \textsuperscript{4} & $R_{1/2}$\textsuperscript{5} & $R_{\rm{c}}$\textsuperscript{8} & $V_{\rm{c}}$\textsuperscript{9} & $\sigma_{\rm{v}}$\textsuperscript{10} & $\chi^2$ \textsuperscript{11} \\
ID & & ($\si{M_{\odot} \, yr^{-1}}$) & ($\si{M_{\odot}}$) & ($\si{"}$) & ($\si{px}$) & ($\si{km/s}$) & ($\si{km/s}$) & \\
\hline



\texttt{30\_d\_77}      & $0.478$ & $-0.4$ & $9.45 $ & $3.51 $ & $4.7 $ & $118.9$ & $33.7$ & $0.72$ \\
\texttt{30\_d\_79}      & $0.828$ & $+0.5$ & $9.17 $ & $3.42 $ & $0.9 $ & $100.4$ & $34.8$ & $1.68$ \\
\texttt{30\_d\_91}      & $0.682$ & $+0.9$ & $9.61 $ & $4.91 $ & $5.6 $ & $154.9$ & $42.4$ & $2.13$ \\
\texttt{30\_d\_108}     & $ 1.19$ & $+1.1$ & $9.73 $ & $1.87 $ & $8.3 $ & $114.5$ & $34.9$ & $1.04$ \\
\texttt{30\_d\_146}     & $1.144$ & $+1.6$ & $9.32 $ & $2.96 $ & $0.9 $ & $130.4$ & $32.2$ & $6.74$ \\
\texttt{30\_d\_148}     & $0.373$ & $-0.3$ & $9.25 $ & $5.45 $ & $14.1$ & $36.7 $ & $17.6$ & $1.15$ \\
\texttt{30\_d\_154}     & $1.216$ & $+1.2$ & $9.85 $ & $3.14 $ & $0.9 $ & $190.1$ & $24.0$ & $4.34$ \\
\texttt{30\_d\_27}      & $1.361$ & $+1.2$ & $9.55 $ & $2.8  $ & $0.9 $ & $71.3 $ & $39.7$ & $0.62$ \\
\texttt{30\_d\_139}     & $0.701$ & $-0.1$ & $8.92 $ & $2.39 $ & $1.0 $ & $109.2$ & $22.3$ & $1.40$ \\
\texttt{32-M1\_375}     & $0.342$ & $-0.0$ & $8.7  $ & $2.94 $ & $11.5$ & $242.3$ & $5.3 $ & $0.29$ \\
\texttt{32-M1\_429}     & $1.104$ & $+1.1$ & $9.91 $ & $3.18 $ & $0.8 $ & $186.8$ & $26.6$ & $3.42$ \\
\texttt{32-M1\_431}     & $0.377$ & $-1.2$ & $8.37 $ & $1.76 $ & $4.2 $ & $37.5 $ & $9.5 $ & $0.39$ \\
\texttt{32-M1\_455}     & $ 0.55$ & $+0.6$ & $10.11$ & $3.43 $ & $15.5$ & $216.6$ & $39.9$ & $1.31$ \\
\texttt{32-M1\_376}     & $0.914$ & $+0.5$ & $9.41 $ & $1.92 $ & $1.6 $ & $167.1$ & $26.9$ & $0.73$ \\
\texttt{32-M1\_414}     & $0.847$ & $-0.2$ & $8.54 $ & $2.12 $ & $4.4 $ & $102.4$ & $34.9$ & $1.30$ \\
\texttt{32-M2\_209}     & $ 0.49$ & $+0.2$ & $9.22 $ & $4.66 $ & $1.0 $ & $148.5$ & $14.2$ & $2.85$ \\
\texttt{32-M2\_216}     & $0.455$ & $+0.7$ & $9.57 $ & $3.98 $ & $5.0 $ & $223.5$ & $27.9$ & $1.45$ \\
\texttt{32-M2\_232}     & $0.321$ & $+0.2$ & $9.33 $ & $5.17 $ & $8.7 $ & $123.4$ & $12.6$ & $5.02$ \\
\texttt{32-M2\_309}     & $0.833$ & $+0.7$ & $9.56 $ & $2.2  $ & $0.9 $ & $166.3$ & $45.1$ & $1.83$ \\
\texttt{32-M2\_154}     & $1.445$ & $+1.0$ & $9.26 $ & $1.92 $ & $4.1 $ & $88.9 $ & $33.6$ & $0.93$ \\
\texttt{32-M3\_32}      & $0.476$ & $+0.4$ & $9.95 $ & $2.56 $ & $3.3 $ & $179.5$ & $24.9$ & $1.14$ \\
\texttt{32-M3\_196}     & $0.322$ & $+0.1$ & $9.78 $ & $3.7  $ & $5.5 $ & $193.9$ & $23.8$ & $2.63$ \\
\texttt{34\_d\_25}      & $0.717$ & $+0.7$ & $9.73 $ & $3.54 $ & $3.0 $ & $194.1$ & $20.2$ & $3.74$ \\
\texttt{34\_d\_50}      & $0.514$ & $+0.1$ & $9.23 $ & $3.25 $ & $2.0 $ & $90.9 $ & $24.9$ & $1.58$ \\
\texttt{34\_d\_90}      & $0.306$ & $-0.9$ & $8.76 $ & $3.01 $ & $7.9 $ & $115.5$ & $4.6 $ & $0.64$ \\
\texttt{34\_d\_103}     & $0.348$ & $-0.2$ & $8.54 $ & $2.31 $ & $4.4 $ & $106.9$ & $11.7$ & $1.83$ \\
\texttt{34\_d\_123}     & $1.006$ & $+0.7$ & $9.44 $ & $3.05 $ & $5.2 $ & $176.0$ & $22.3$ & $3.57$ \\
\texttt{34\_d\_144}     & $0.321$ & $-0.8$ & $8.07 $ & $2.99 $ & $15.5$ & $359.3$ & $38.6$ & $0.82$ \\
\texttt{34\_d\_154}     & $1.305$ & $+1.8$ & $10.37$ & $2.51 $ & $7.5 $ & $196.9$ & $53.2$ & $3.22$ \\
\texttt{34\_d\_155}     & $1.027$ & $+1.5$ & $10.26$ & $2.94 $ & $2.9 $ & $197.2$ & $72.5$ & $11.44$ \\
\texttt{34\_d\_17}      & $1.026$ & $+0.1$ & $8.77 $ & $1.97 $ & $3.8 $ & $172.2$ & $25.2$ & $2.29$ \\
\texttt{51\_28}         & $  0.7$ & $+0.7$ & $9.33 $ & $2.24 $ & $11.2$ & $218.7$ & $23.1$ & $5.12$ \\
\texttt{51\_123}        & $1.245$ & $+1.0$ & $9.96 $ & $1.79 $ & $6.1 $ & $205.8$ & $50.0$ & $0.76$ \\
\texttt{51\_23}         & $0.925$ & $-0.1$ & $8.77 $ & $2.0  $ & $0.9 $ & $52.2 $ & $33.1$ & $0.96$ \\
\texttt{51\_49}         & $0.912$ & $+0.1$ & $8.78 $ & $2.15 $ & $0.9 $ & $72.6 $ & $18.4$ & $0.52$ \\
\texttt{51\_77}         & $0.727$ & $-0.3$ & $8.65 $ & $2.11 $ & $1.0 $ & $92.0 $ & $10.4$ & $3.64$ \\
\texttt{61\_7}          & $1.172$ & $+1.1$ & $8.98 $ & $2.43 $ & $3.0 $ & $171.4$ & $24.6$ & $4.59$ \\
\texttt{61\_12}         & $0.339$ & $-0.3$ & $8.16 $ & $2.43 $ & $1.0 $ & $31.1 $ & $37.4$ & $1.65$ \\
\texttt{61\_25}         & $0.329$ & $+0.1$ & $9.59 $ & $4.48 $ & $1.0 $ & $82.5 $ & $22.8$ & $1.20$ \\
\texttt{61\_39}         & $0.432$ & $+0.1$ & $9.19 $ & $2.39 $ & $1.0 $ & $105.1$ & $21.1$ & $0.99$ \\
\texttt{61\_87}         & $1.163$ & $+1.4$ & $10.41$ & $2.53 $ & $0.9 $ & $238.1$ & $40.6$ & $2.69$ \\
\texttt{61\_102}        & $0.465$ & $-0.4$ & $8.85 $ & $1.89 $ & $15.5$ & $115.6$ & $11.4$ & $0.89$ \\
\texttt{61\_117}        & $0.599$ & $-0.3$ & $8.56 $ & $3.14 $ & $7.2 $ & $193.9$ & $14.9$ & $0.68$ \\
\texttt{61\_79}         & $0.883$ & $+0.1$ & $8.82 $ & $2.16 $ & $15.2$ & $670.7$ & $22.7$ & $0.41$ \\
\texttt{79\_41}         & $0.749$ & $+0.9$ & $9.7  $ & $3.72 $ & $3.4 $ & $158.0$ & $35.2$ & $9.12$ \\
\texttt{79\_42}         & $1.458$ & $+1.6$ & $10.2 $ & $3.42 $ & $0.8 $ & $113.7$ & $42.2$ & $15.16$ \\
\texttt{79\_51}         & $0.912$ & $+0.0$ & $9.3  $ & $2.82 $ & $3.7 $ & $124.7$ & $34.5$ & $12.04$ \\


	\end{tabular}
\caption[Morpho-kinematical parameters II]{Table of morphological and kinematical parameters: part II. Description is simliar to Table \ref{table:params1}}
\label{table:params2}
\end{table}


\newpage
\begin{table}[htbp]
	\centering
	\vspace{-250pt}

	\begin{tabular}{c c c c c c c c c c c}

\hline
Galaxy\textsuperscript{1} & $z$\textsuperscript{2} & $\log_{10} \rm{SFR}$\textsuperscript{3} & $\log_{10} M_{*}$ \textsuperscript{4} & $R_{1/2}$\textsuperscript{5} & $R_{\rm{c}}$\textsuperscript{8} & $V_{\rm{c}}$\textsuperscript{9} & $\sigma_{\rm{v}}$\textsuperscript{10} & $\chi^2$ \textsuperscript{11} \\
ID & & ($\si{M_{\odot} \, yr^{-1}}$) & ($\si{M_{\odot}}$) & ($\si{"}$) & ($\si{px}$) & ($\si{km/s}$) & ($\si{km/s}$) & \\
\hline

\texttt{79\_80}         & $0.743$ & $+0.5$ & $9.61 $ & $2.1  $ & $3.0 $ & $186.0$ & $28.4$ & $2.58$ \\
\texttt{79\_112}        & $0.433$ & $-0.1$ & $8.78 $ & $2.22 $ & $2.6 $ & $98.9 $ & $16.1$ & $0.85$ \\
\texttt{79\_114}        & $0.668$ & $+1.3$ & $10.26$ & $4.56 $ & $9.4 $ & $328.5$ & $29.0$ & $33.56$ \\
\texttt{79\_120}        & $0.896$ & $+1.1$ & $9.84 $ & $2.92 $ & $2.6 $ & $192.5$ & $47.1$ & $4.91$ \\
\texttt{79\_69}         & $0.668$ & $-0.1$ & $8.54 $ & $2.58 $ & $15.5$ & $165.6$ & $19.9$ & $0.23$ \\
\texttt{84-N\_8}        & $0.407$ & $+0.6$ & $10.25$ & $4.89 $ & $5.1 $ & $172.6$ & $24.9$ & $1.70$ \\
\texttt{84-N\_20}       & $0.668$ & $+0.9$ & $9.8  $ & $3.0  $ & $7.2 $ & $135.3$ & $20.2$ & $0.92$ \\
\texttt{84-N\_21}       & $0.669$ & $+0.2$ & $9.09 $ & $2.65 $ & $0.9 $ & $18.8 $ & $25.4$ & $0.78$ \\
\texttt{84-N\_66}       & $0.773$ & $-0.1$ & $8.81 $ & $2.4  $ & $4.0 $ & $195.1$ & $20.8$ & $0.82$ \\
\texttt{84\_39}         & $0.665$ & $-0.5$ & $9.29 $ & $2.56 $ & $1.0 $ & $80.4 $ & $27.7$ & $0.91$ \\
\texttt{84\_54}         & $0.506$ & $+0.4$ & $9.36 $ & $3.11 $ & $3.5 $ & $101.6$ & $25.1$ & $2.13$ \\
\texttt{84\_132}        & $0.471$ & $-0.6$ & $9.09 $ & $1.94 $ & $3.4 $ & $51.0 $ & $18.2$ & $1.02$ \\
\texttt{84\_146}        & $0.747$ & $+0.7$ & $9.41 $ & $3.41 $ & $13.2$ & $450.0$ & $24.2$ & $2.85$ \\
\texttt{84\_98}         & $0.444$ & $-0.9$ & $9.06 $ & $16.04$ & $8.4 $ & $117.7$ & $14.2$ & $0.45$ \\



	\end{tabular}
\caption[Morpho-kinematical parameters III]{Table of morphological and kinematical parameters: part III. Description is simliar to Table \ref{table:params1}}
\label{table:params3}
\end{table}


















\end{document}